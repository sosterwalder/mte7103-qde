% -*- mode: latex; coding: utf-8 -*-

\section{Approach}

\newthought{To reach the intended goal,} an existing draft of a software
architecture was continued, refined and completed. The draft was the result of
the preceding project work,~\citetitle{osterwalder-qde-2016}. The software
itself was developed using literate programming and the agile methodology
extreme programming. For rendering the sphere tracing algorithm was used, which
was established in the preceding project
work,~\citetitle{osterwalder-volume-2016}.

\subsection{Literate programming}

\newthought{Documenting software is crucial.} All too frequently however,
software is not documented properly or the documentation is even neglected as it
can be quite costly with seemingly little benefit. Several examples show
however, that documenting software is crucial. No documentation at all, outdated
or irrelevant documentation can lead to unforeseen efforts concerning time and
costs. To prevent such unforeseen efforts the developed software and this thesis
was written with a paradigm called~\emph{literate programming}. The paradigm was
introduced in~\citeyear{knuth-lp-1984} by~\citeauthor{knuth-lp-1984}. It
proposes to consider programs to be works of literature and to explain to human
beings what the computer shall do to achieve certain goals.

% \begin{figure}[h]
%   \begin{flushleft} \small
% \begin{minipage}{\linewidth}\label{scrap3}\raggedright\small
% \NWtarget{load-node-defs}{} $\langle\,${\itshape Load node definitions}\nobreak\ {\footnotesize {1}}$\,\rangle\equiv$
% \vspace{-1ex}
% \begin{pythoncode}
% def load_node_definitions(self):
%     """Loads all files with the ending
%     NODES_EXTENSION within the NODES_PATH
%     directory, relative to the current working directory.
%     """
% 
%     if os.path.exists(self.nodes_path):
%         |\hbox{$\langle\,${\itshape Find and load $\cdots$}$\,\rangle$}|
%     else:
%         |\hbox{$\langle\,${\itshape Output warning, that $\cdots$}$\,\rangle$}||\NWsep|
% |\NWsep|
% \end{pythoncode}
% \end{minipage}%\vspace{4ex}
% \end{flushleft}
% \caption{Example of a code fragment for loading node definition files. This
%   fragment is itself part of another fragment which generates a python file.}
% \label{lst:literate-program}
% \end{figure}

\subsection{Agile software development}

\newthought{Change is one of the main challenges} when developing software. Be
it induced by new requirements or by newly-gained insights during development.
Traditional software engineering methodologies, such as the waterfall model or
incremental development, struggle with change. By applying basic principles,
agile development methodologies try to overcome this problem. These principles
may vary depending on the used methodology, but the fundamental principles are:
\begin{enumerate*}
  \item rapid feedback,
  \item assume simplicity,
  \item incremental change,
  \item embracing change and
  \item quality work.
\end{enumerate*}% ~\cite{beck-xp-2004}

\newthought{An adapted version of extreme programming} was used for this thesis.
This methodology was chosen as after the preceding project
work,~\citetitle{osterwalder-qde-2016}, several things were still subject to
change and therefore an exact planning, analysis and design, as traditional
methodologies require it, would not have been very practical.