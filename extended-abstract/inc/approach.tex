% -*- mode: latex; coding: utf-8 -*-

\section{Approach}

\newthought{To reach the intended goal,} an existing draft of a software
architecture was continued, refined and completed. The draft was the result of
the preceding project work. The software itself was developed using literate
programming and the agile methodology extreme programming.

\newthought{Documenting software is crucial.} All too frequently however,
software is not documented properly or the documentation is even neglected as it
can be quite costly with seemingly little benefit. But no documentation at all,
outdated or irrelevant documentation can lead to unforeseen efforts concerning
time and costs. To prevent such unforeseen efforts the developed software and
this thesis was written with a paradigm called~\emph{literate programming}. The
paradigm was introduced in~\citeyear{knuth-lp-1984}
by~\citeauthor{knuth-lp-1984}. It proposes to consider programs to be works of
literature and to explain to human beings what the computer shall do to achieve
certain goals. To overcome one of the main challenges when developing the software
--- change --- an adapted version of extreme programming was used. This
methodology was chosen as after the preceding project work several things were
still subject to change and therefore an exact planning, analysis and design, as
traditional methodologies require it, would not have been very practical.