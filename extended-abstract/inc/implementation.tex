% -*- mode: latex; coding: utf-8 -*-

\section{Implementation}

\newthought{The results of this thesis} are an architecture for a software
program, and the program itself, written using the literate programming
paradigm.
\newthought{Three aspects define the software architecture}:
\begin{enumerate*}
  \item the model-view-view model software design pattern using controllers in
    addition,
  \item the layered software architectural pattern and
  \item the observer software design pattern, allowing communication between
    components of the software.
\end{enumerate*}
\newthought{The software itself} is an editor which allows~\emph{modeling}
objects, ~\emph{composing} objects to scenes and~\emph{rendering} scenes in
real-time. Scenes are stored in a scene graph structure and are represented by a
tree view. Each scene can contain one or more objects, defined by external
files and represented as nodes in the graph structure. The parameters of
objects are used for interconnections between nodes. For rendering, the sphere
tracing algorithm established in the preceding project work was used.

% The editor was written in the python programming language
% using the literate programming paradigm and uses the Qt framework as basis and
% for the graphical user interface as well as OpenGL for rendering.

% The graph structure allows adding, removing and connecting nodes to
% complex objects. Every node has one ore more parameters, such as size or
% position (of an object).

% \newthought{Compositing} includes two aspects: the
% composition of objects into scenes and the composition of an animation which is
% defined by multiple scenes which follow a chronological order. The first aspect
% is realized by a scene graph structure, which contains at least a root scene.
% Each scene may contain objects in form of nodes which can be connected. The
% second aspect is realized by a time line, which allows a chronological
% organization of scenes.

% \newthought{For rendering} a highly optimized algorithm
% based on ray tracing is used. The algorithm is called sphere tracing and allows
% the rendering of ray traced scenes in real time on the GPU. Real-time means in
% this context being able to generate at least 25 images in one second. Contingent
% upon the used rendering algorithm all models are modeled using implicit
% surfaces. For rendering the sphere tracing algorithm was used, which was
% established in the preceding project work,~\citetitle{osterwalder-volume-2016}.
