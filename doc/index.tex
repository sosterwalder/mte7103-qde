\newcommand{\NWtarget}[2]{\hypertarget{#1}{#2}}
\newcommand{\NWlink}[2]{\hyperlink{#1}{#2}}
\newcommand{\NWtxtMacroDefBy}{Fragment defined by}
\newcommand{\NWtxtMacroRefIn}{Fragment referenced in}
\newcommand{\NWtxtMacroNoRef}{Fragment never referenced}
\newcommand{\NWtxtDefBy}{Defined by}
\newcommand{\NWtxtRefIn}{Referenced in}
\newcommand{\NWtxtNoRef}{Not referenced}
\newcommand{\NWtxtFileDefBy}{File defined by}
\newcommand{\NWtxtIdentsUsed}{Uses:}
\newcommand{\NWtxtIdentsNotUsed}{Never used}
\newcommand{\NWtxtIdentsDefed}{Defines:}
\newcommand{\NWsep}{${\diamond}$}
\newcommand{\NWnotglobal}{(not defined globally)}
\newcommand{\NWuseHyperlinks}{}
% -*- mode: latex; coding: utf-8 -*-

\documentclass[
    a4paper,      % paper format
    10pt,         % fontsize
    %twoside,     % double-sided
    openright,    % begin new chapter on right side
    notitlepage,  % use no standard title page
    parskip=half, % set paragraph skip to half of a line
]{scrreprt}       % KOMA-script report
% ]{tufte-book}
%---------------------------------------------------------------------------
\raggedbottom{}
\KOMAoptions{cleardoublepage=plain} % Add header and footer on blank pages


% Load Standard Packages:
%---------------------------------------------------------------------------
\usepackage{scrpage2}                    % Control of page headers and footers in LaTeX,
\usepackage{marginnote}
                                         % needed for e.g. deftripstyle (to defined page styles)
\usepackage[english]{babel}              % English hyphenation
\usepackage[utf8]{inputenc}              % UTF-8 input encoding
\usepackage[T1]{fontenc}                 % hyphenation of words with ä,ö and ü
\usepackage{textcomp}                    % additional symbols
\usepackage{float}                       % floating objects
\usepackage{booktabs,tabularx}           % package for nicer tables
\usepackage{tocvsec2}                    % provides means of controlling the sectional numbering
\usepackage{pgfgantt}                    % Provides GANTT charts
\usepackage[owncaptions]{vhistory}       % Provides framework for creating history outline
\renewcommand{\vhhistoryname}{Versions}  % Rename version history to german name "Versionen"
\usepackage{csquotes}                    % Quotes
\usepackage{nameref}                     % Allows referencing of names
\usepackage{blindtext}                   % Dummy text
%---------------------------------------------------------------------------

% Environments
% ---------------------------------------------------------------------------
\newenvironment{loggentry}[2]% date, heading
{\noindent\textbf{#1}\marginnote{#2}\\}

% Bibliography
%---------------------------------------------------------------------------
% \usepackage[
%     style=alphabetic,
%     backend=biber,
%     citestyle=authoryear
% ]{biblatex}
% \addbibresource{inc/static/bibliography.bib}
\usepackage[
    backend=biber,
    style=ieee,
    sortlocale=de_DE,
    natbib=true,
    url=false, 
    doi=true,
    eprint=false
]{biblatex}
% \bibliographystyle{IEEEtranS}
\addbibresource{inc/bibliography.bib}
\DefineBibliographyStrings{ngerman}{
    andothers = {{et\,al\adddot}},
}
%---------------------------------------------------------------------------

% Load Math Packages
%---------------------------------------------------------------------------
\usepackage{mathtools}                       % Provide equation and gather environments
\usepackage{amsthm}                          % Provide the possibility to define definitions
\theoremstyle{definition}                    % Add new theorem style
\newtheorem{definition}{Definition}[section] % Add new theorem
\usepackage{bm}                              % bold font in math mode
\usepackage{amssymb}                         % mathematical special characters, e.g. mathbb
\usepackage{exscale}                         % mathematical size corresponds to textsize
\usepackage{esvect}                          % Provides nicer vector display in math mode
%---------------------------------------------------------------------------

% Definition of fonts
%---------------------------------------------------------------------------
\DeclareFixedFont{\ttb}{T1}{txtt}{bx}{n}{9} % for bold
\DeclareFixedFont{\ttm}{T1}{txtt}{m}{n}{9}  % for normal
%---------------------------------------------------------------------------

% Definition of colors
%---------------------------------------------------------------------------
\RequirePackage{color}
\definecolor{linkblue}{rgb}{0,0,0.8}       % Standard
\definecolor{darkblue}{rgb}{0,0.08,0.45}   % Dark blue
\definecolor{bfhgrey}{rgb}{0.41,0.49,0.57} % BFH grey
\definecolor{linkcolor}{rgb}{0,0,0}
\colorlet{Black}{black}
\definecolor{keywords}{rgb}{255,0,0}
\definecolor{red}{rgb}{0.6,0,0}
\definecolor{green}{rgb}{0,0.5,0}
\definecolor{blue}{rgb}{0,0,0.5}
% Syntax colors
\definecolor{syntaxRed}{rgb}{0.6,0,0}
\definecolor{syntaxBlue}{rgb}{0,0,0.5}
\definecolor{syntaxComment}{rgb}{0,0.5,0}
% Background colors
\definecolor{syntaxBackground}{rgb}{0.95, 0.95, 0.95}

% Load listings package
% which provides source code formatting
%---------------------------------------------------------------------------
\usepackage{listings}
\lstdefinestyle{python}{%
    language=Python,
    basicstyle=\ttm\ttfamily\linespread{1.15},
    backgroundcolor = \color{syntaxBackground},
    % columns=fullflexible,
    commentstyle=\color{green},
    emphstyle=\ttb\color{red},
    escapechar=§,
    frame=tlbr,
    framesep=0.2cm,
    framerule=0pt,
    numbers=left,
    numbersep=5pt,                   % how far the line-numbers are from the code
    numberstyle=\tiny\color{gray}, % the style that is used for the line-numbers
    identifierstyle=\color{black},
    keywordstyle=\ttb\color{blue},
    otherkeywords={self, param},
    % procnamekeys={def,class},
    showspaces=false,
    showstringspaces=false,
    showtabs=false,
    stringstyle=\color{syntaxComment},
    tab=\rightarrowfill,
    xleftmargin=0.7cm,
}
\lstset{style=python}
% Hyperref Package (Create links in a pdf)
%---------------------------------------------------------------------------
\usepackage[
    ngerman,bookmarks,plainpages=false,pdfpagelabels,
    backref = {false},                                        % No index backreference
    colorlinks = {true},                                      % Color links in a PDF
    hypertexnames = {true},                                   % no failures "same page(i)"
    bookmarksopen = {true},                                   % opens the bar on the left side
    bookmarksopenlevel = {0},                                 % depth of opened bookmarks
    pdftitle = {QDE --- A visual animation system},           % PDF-property
    pdfauthor = {Sven Osterwalder},                           % PDF-property
    pdfsubject = {QDE},                                       % PDF-property
    linkcolor = {linkcolor},                                  % Color of Links
    citecolor = {linkcolor},                                  % Color of Cite-Links
    urlcolor = {linkcolor},                                   % Color of URLs
]{hyperref}

% Geometry package: Set up page dimension
%---------------------------------------------------------------------------
\usepackage[a4paper,
    left=25mm,
    right=25mm,
    top=27mm,
    headheight=20mm,
    headsep=10mm,
    textheight=242mm,
    footskip=15mm
]{geometry}

% Makeindex Package
%---------------------------------------------------------------------------
\usepackage{makeidx}
\makeindex

% Glossary Package
%---------------------------------------------------------------------------
\usepackage[nonumberlist,nomain]{glossaries}
% -*- coding: utf-8 -*-\makeglossaries{}

% Fancyrb package
%---------------------------------------------------------------------------
\usepackage{fancyvrb}
\RecustomVerbatimCommand{\VerbatimInput}{VerbatimInput}
{fontsize=\footnotesize,
    frame=lines,  % top and bottom rule only
    framesep=2em, % separation between frame and text rulecolor=\color{Gray},
    label=\fbox{\color{Black}},
    labelposition=topline,
    % commandchars=\|\(\), % escape character and argument delimiters for
    % commands within the verbatim
    % commentchar=*        % comment character
}

% TODO notes package
%---------------------------------------------------------------------------
\usepackage[textwidth=65mm]{todonotes}

\begin{document}
\settocdepth{section}
\pagenumbering{roman}

% Title variables
%---------------------------------------------------------------------------
\providecommand{\titletext}{QDE --- A visual animation system.}
\providecommand{\subtitletext}{MTE7103}
\providecommand{\subsubtitletext}{Master-Thesis}

% Set up header and footer using page style
%---------------------------------------------------------------------------
\deftripstyle{newlayout}
  [0pt] % no header line
  [0pt] % no footer line
  {} % Header left
  {} % Header center
  {} % Header right
  {\color{bfhgrey} \footnotesize \titletext, Version \vhCurrentVersion,
      \vhCurrentDate} % Footer left
  {} % Footer center
  {\color{bfhgrey} \thepage} % Footer right

\deftripstyle{titlepageStyle}
  [0pt] % no header line
  [0pt] % no footer line
  {} % Header left
  {} % Header center
  {} % Header right
  {\color{bfhgrey}\fontsize{9pt}{10pt}\selectfont
    Berner Fachhochschule | Haute école spécialisée bernoise | Bern
    University of Applied Sciences} % Footer left
  {} % Footer center
  {} % Footer right


\pagestyle{newlayout}
% use "pagestyle" also on chapter starting pages
\renewcommand{\chapterpagestyle}{newlayout}
\renewcommand{\chaptermark}[1]{\markboth{\thechapter.  #1}{}}
\renewcommand*{\headfont}{\normalfont}
\renewcommand*{\footfont}{\normalfont}

% Title Page and Abstract
%---------------------------------------------------------------------------
\setcounter{page}{1}
% -*- mode: latex; coding: utf-8 -*-

\begin{titlepage}

    % BFH-Logo absolute placed at (28,12) on A4 and picture (16:9 or 15cm x 8.5cm)
    % Actually not a realy satisfactory solution but working.
    %---------------------------------------------------------------------------
    \setlength{\unitlength}{1mm}
    % \includegraphics[scale=1.0]{img/BFH_Logo_B}
    BFH Logo

    \begin{picture}(150,2)
        \put(0,0){\color{bfhgrey}\rule{150mm}{2mm}}
    \end{picture}

    \begin{figure}[H]
        \hspace*{0.25cm}
        % \includegraphics{img/logo.pdf}
        LOGO
    \end{figure}

    \begin{picture}(150,2)
        \put(0,0){\color{bfhgrey}\rule{150mm}{2mm}}
    \end{picture}

    \begin{flushleft}
        \fontsize{26pt}{28pt}\selectfont
        \textbf{\titletext} \\
        \vspace{3mm}
        \subtitletext{}\\
        \vspace{6mm}
        \fontsize{14pt}{16pt}\selectfont
        \textbf{\subsubtitletext} \\
        \vspace{3mm}

        \fontsize{10pt}{17pt}\selectfont
        \begin{tabbing}
        xxxxxxxxxxxxxxx   \= xxxxxxxxxxxxxxxxxxxxxxxxxxxxxxxxxxxxxxxxxxxxxxx \kill
        Major:            \> Computer science                                         \\
        Author:           \> Sven Osterwalder\protect\footnotemark[1]{}         \\
        Advisor:          \> Prof.~Claude Fuhrer\protect\footnotemark[2]{} \\
        Expert:           \> Dr.~Eric Dubuis\protect\footnotemark[3]{} \\
        Date:             \> \vhCurrentDate{}\\
        Version:          \> \vhCurrentVersion\\
        \end{tabbing}
    \end{flushleft}
    \footnotetext[1]{sven.osterwalder@students.bfh.ch}
    \footnotetext[2]{claude.fuhrer@bfh.ch}
    \footnotetext[3]{eric.dubuis@comet.ch}

    \vfill
    % \includegraphics[height=\baselineskip]{img/by-sa}\\ \small{\sffamily{Licensed under the Creative Commons Attribution-ShareAlike 3.0 License}}
    by-sa

    \thispagestyle{titlepageStyle}

\end{titlepage}
% -*- mode: latex; coding: utf-8 -*-

% Versions:
% -----------------------------------------------

\chapter*{}
\label{chap:versions}

\begin{versionhistory}
    \vhEntry{0.1}{29.03.2017}{SO}{Initial creation of the documentation}
\end{versionhistory}
\listoftodos{}
\addcontentsline{toc}{chapter}{Abstract}
% -*- coding: utf-8 -*-

\chapter*{Abstract}
\label{chap:abstract}

\blindtext
% Table of contents
%---------------------------------------------------------------------------
\tableofcontents
\cleardoublepage{}

% Main part
%---------------------------------------------------------------------------
\pagenumbering{arabic}
% -*- mode: latex; coding: utf-8 -*-

\chapter{Introduction}
\label{chap:introduction}

\blindtext{}
\blindtext{}

\section{Purpose and situation}
\label{sec:purpose}

\subsection{Motivation}
\label{subsec:motivation}

\blindtext{}

\subsection{Objectives and limitations}
\label{subsec:objectives}

\blindtext{}

\subsection{Preliminary activities}
\label{subsec:preliminary}

\blindtext{}

\section{Related works}
\label{sec:related-works}

Preliminary to this thesis two project works were done: ``Volume
ray casting --- basics \& principles''~\cite{osterwalder_volume_2016}, which
describes the basics and principles of sphere tracing, a special form of ray
tracing, and ``QDE --- a visual animation system,
architecture''~\cite{osterwalder_qde_2016}, which established the ideas and
notions of an editor and a player component as well as the basis for a possible
software architecture for these components. The latter project work is presented
in detail in the chapter about the procedure, the former project work is
presented in the chapter about the implementation.

\section{Document structure}
\label{sec:document-structure}

This document is divided into N chapters, the first being this introduction. The
second chapter on \textit{administrative aspects} shows the planning of the
project, including the involved persons, deliverables and the phases and
milestones.

The administrative aspects are followed by a chapter on the \textit{procedure}.
The purpose of that chapter is to show the procedure concerning the execution of
this thesis. It introduces a concept called literate programming, which builds
the foundation for this thesis. Furthermore it establishes a framework for the
actual implementation, which is heavily based on the previous project work,
``QDE --- a visual animation system, architecture''~\cite{osterwalder_qde_2016}
and also includes standards and principles.

The following chapter on the \textit{implementation} shows how the
implementation of the editor and the player component as well as how the
rendering is done using a special form of ray tracing as described in ``Volume
ray casting --- basics \& principles''~\cite{osterwalder_volume_2016}. As the
editor component defines the whole data structure it builds the basis of the
thesis and can be seen as main part of the thesis. The player component re-uses
concepts established within the editor.

Given that literate programming is very complete and elaborated, as components
being developed using this procedure are completely derived from the
documentation, the actual implementation is found in the appendix as otherwise
this thesis would be simply too extensive.

The last chapter is \textit{discussion and conclusion} and discusses the
procedure as well as the implementation. Some further work on the editor and the
player components is proposed as well.

After the regular content follows the \textit{appendix}, containing the
requirements for building the before mentioned components, the actual source
code in form of literal programming as well as test cases for the components.% -*- mode: latex; coding: utf-8 -*-

\chapter{Administrative aspects}
\label{chap:administrative_aspects}

Some administrative aspects of this thesis are covered, while they are not
required for the understanding of the result.

The whole documentation uses the male form, whereby both genera are equally
meant.

\section{Involved persons}
\label{sec:involved_persons}

\begin{table}[h]
  \begin{tabularx}{\textwidth}{|l|l|X|}
    \textbf{Author}  & Sven Osterwalder\protect\footnotemark[1]{}     & \\
    \textbf{Advisor} & Prof.\ Claude Fuhrer\protect\footnotemark[2]{} & \textit{Supervises the student doing the thesis}\\
    \textbf{Expert}  & Dr.\ Eric Dubuis\protect\footnotemark[3]{}     & \textit{Provides expertise concerning the thesis's subject, monitors and grades the thesis}\\
  \end{tabularx}
  \caption{List of the involved persons.}
\end{table}
\footnotetext[1]{sven.osterwalder@students.bfh.ch}
\footnotetext[2]{claude.fuhrer@bfh.ch}
\footnotetext[3]{eric.dubuis@comet.ch}

\section{Deliverables}
\label{sec:deliverables}

\begin{itemize}
\item \textbf{Report} \\
  \blindtext{}
\item \textbf{Implementation} \\
  \blindtext{}
\end{itemize}

\section{Organization of work}
\label{sec:organization-of-work}

\subsection{Meetings}
\label{subsec:meetings}

Various meetings with the supervising professor, Mr. Claude Fuhrer, helped
reaching the defined goals and preventing erroneous directions of the thesis.
The supervisor supported the author of this thesis by providing suggestions
throughout the held meetings. The minutes of the meetings may be found under
<<Meeting minutes>>.

\subsection{Phases of the project and milestones}
\label{subsec:project-phases-milestones}

\begin{table}[h]
  \begin{tabularx}{\textwidth}{|X|X|r|}
    \hline{}
    \textbf{Phase}   & \textbf{Description} & \textbf{Week / 2017} \\
    \hline{}
    Start of the project & & 8 \\
    Definition of objectives and limitation & & 8-9 \\
    Documentation and development & & 8-30 \\
    Corrections & & 30-31 \\
    Preparation of the thesis' defense & & 31-32 \\
    \hline
  \end{tabularx}
  \caption{Phases of the project.}
\end{table}

\begin{table}[h]
  \begin{tabularx}{\textwidth}{|X|X|r|}
    \hline{}
    \textbf{Phase}   & \textbf{Description} & \textbf{End of week / 2017} \\
    \hline{}
    Project structure is set up & & 8 \\
    Mandatory project goals are reached & & 30 \\
    Hand-in of the thesis & & 31 \\
    Defense of the thesis & & 32 \\
    \hline
  \end{tabularx}
  \caption{Milestones of the project.}
\end{table}

\begin{figure}[H]
    \begin{ganttchart}[
        vgrid,
        x unit=0.5cm,
        bar/.append style={fill=bfhgrey!50},
    ]{1}{26}
        \gantttitle{2017}{26} \ganttnewline{}
        \gantttitlelist{7,...,32}{1} \ganttnewline{}
        \ganttbar{Start of the project}{1}{1} \ganttnewline{}
        \ganttmilestone{Project is set up}{1} \ganttnewline{}
        \ganttlinkedbar{Objectives and limitations}{2}{3} \ganttnewline{}
        \ganttlinkedbar{Documentation}{3}{23} \ganttnewline{}
        \ganttbar{Development}{3}{23} \ganttnewline{}
        \ganttmilestone{Goals reached}{23} \ganttnewline{}
        \ganttlinkedbar{Corrections}{23}{24} \ganttnewline{}
        \ganttmilestone{Hand-in}{24} \ganttnewline{}
        \ganttlinkedbar{Thesis' defense preparation}{25}{26} \ganttnewline{}
        \ganttmilestone{Thesis defense}{26}
    \end{ganttchart}
    \caption{Schedule of the project by calendar weeks, including milestones.}\label{fig:timeschedule}
\end{figure}% -*- mode: latex; coding: utf-8 -*-

\chapter{Procedure}
\label{chap:procedure}

\subsection{Literate programming}
\label{subsec:literate-programming}

This thesis' implementation is done by a procedure named ``literate
programming'', invented by Donald Knuth. What this means, is that the
documentation as well as the code for the resulting program reside in the same
file. The documentation is then /weaved/ into a separate document, which may be
any by the editor support format. The code of the program is /tangled/ into a
run-able computer program.~\todo[inline]{Provide more information about literate
programming. Citations, explain fragments, explain referencing fragments, code
structure does not have to be ``normal''}

Originally it was planned to develop this thesis' application test driven,
providing (unit-) test-cases first and implementing the functionality
afterwards. Initial trails showed quickly that this method, in company with
literate programming, would exaggerate the effort needed. Therefore conventional
testing is used. Test are developed after implementing functionality and run
separately. A coverage as high as possible is intended. Test cases are /tangled/
too, and may be found in the appendix.\todo[inline]{Insert reference/link to test cases here.}

\section{Standards and principles}
\label{sec:standards-principles}

\subsection{Requirements}
\label{subsec:requirements}

The requirements are defined by the preceding project work,~\enquote{QDE --- a
  visual animation system, software architecture}~\citep[p. 8
ff.]{osterwalder_qde_2016}, and are still valid.

For the editor application however, Python is used as a programming language.
This decision is made as the author of the thesis has several years of
experience concerning Python and as the performance of the editor is not
a critical factor. By performance all aspects are concerned, e.g. the evaluation
of the node graph or rendering itself.

As Python provides no direct bindings to Qt, an additional library is needed,
which provides those bindings. Currently there exist two Python bindings for Qt:
PySide and PyQt. As Qt version 5 is used, the bindings need to provide access to
version 5 too. Currently this is only achieved by PyQt5 in a stable and complete
way. PySide2 supports Qt version 5 too, is although under heavy development and
far from being complete and stable.

Therefore PyQt5 is an additional requirement.

\subsection{Code}
\label{subsec:code}

\begin{itemize}
\item Classes use camel case.
\item Folders / name-spaces use only small letters.
\item Methods are all small caps and use underscores as spaces.
\item Signals: do\_something
\item Slots: on\_something
\item Importing: {{{verb(from Foo import Bar)}}}\\
      As the naming of the PyQt5 modules prefixes them by /Qt/, it is very
      unlikely to have naming conflicts with other modules. Therefore the import
      format {{{verb(from PyQt5 import [QtModuleName])}}} is used. This still
      provides a (relatively) unique naming most probably without any conflicts
      but reduces the effort when writing a bit. The import of system modules is
      therefore as follows.
\end{itemize}

\subsubsection{Layering}
\label{ssubsec:layering}

Concerning the architecture, a layered architecture is foreseen, as stated in
\cite[p. 38 ff.]{osterwalder_qde_2016}. A relaxed layered architecture leads to
low coupling, reduces dependencies and enhances cohesion as well as clarity.

As the architecture's core \todo{Link to components} components are all graphical, a graphical user
interface for those components is developed. As the their data shall be
exportable, it would be relatively tedious if the graphical user interface would
hold and control that data. Instead models and model-view separation are used.
Additionally controllers are introduced which act as workflow objects of the
=application= layer and interfere between the model and its view.

\subsubsection{Model-View-Controller}
\label{ssubsec:mvc}

While models may be instantiated anywhere directly, this would although not
contribute to having clean code and sane data structures. Instead controllers,
lying within the {{{verb(application)}}} layer, will manage instances of models.
The instantiating may either be induced by the graphical user interface
or by the player when loading and playing exported animations.

A view may never contain model-data (coming from the {{{verb(domain)}}} layer)
directly, instead view models are used \cite{martin_fowler_presentation_2004}.

The behavior described above corresponds to the well-known model-view-controller
pattern expanded by view models.

As Qt is used as the core for the editor, it may be quite obvious to use Qt's
model/view programming practices, as described by
[fn:20:http://doc.qt.io/qt-5/model-view-programming.html]. However, Qt combines
the controller and the view, meaning the view acts also as a controller while
still separating the storage of data. The editor application does not actually
store data (in a conventional way, e.g. using a database) but solely exports it.
Due to this circumstance the model-view-controller pattern is explicitly used,
as also stated in \cite[p. 38]{osterwalder_qde_2016}.

\todo[inline]{Describe the exact process of communication between ViewModel,
Controller and Model.}

To avoid coupling and therefore dependencies, signals and
slots[fn:16:http://doc.qt.io/qt-5/signalsandslots.html] are used in terms of the
observer pattern to allow inter-object and inter-layer communication.% -*- mode: latex; coding: utf-8 -*-

\chapter{Implementation}
\label{chap:implementation}

% -*- mode: latex; coding: utf-8 -*-

\section{Editor}
\label{sec:editor}

% -*- mode: latex; coding: utf-8 -*-

\section{Player}
\label{sec:player}

% -*- mode: latex; coding: utf-8 -*-

\section{Rendering}
\label{sec:rendering}

% Glossary
%---------------------------------------------------------------------------
\cleardoublepage{}
\phantomsection{}
\addcontentsline{toc}{chapter}{Glossary}
\glsaddall{}
\printglossaries{}

% Bibliography
%---------------------------------------------------------------------------
\cleardoublepage{}
\phantomsection{}
\addcontentsline{toc}{chapter}{Bibliography}
\printbibliography{}

% Listings
%---------------------------------------------------------------------------
%\cleardoublepage
\phantomsection{}
\addcontentsline{toc}{chapter}{List of figures}
\listoffigures
%\cleardoublepage
\phantomsection{}
\addcontentsline{toc}{chapter}{List of tables}
\listoftables
%\cleardoublepage
\phantomsection{}
\addcontentsline{toc}{chapter}{List of listings}
\lstlistoflistings{}

% Index
%---------------------------------------------------------------------------
%\cleardoublepage
%\phantomsection{}
%\addcontentsline{toc}{chapter}{Stichwortverzeichnis}
%\renewcommand{\indexname}{Stichwortverzeichnis}
%\printindex

% Appendix
%---------------------------------------------------------------------------
% -*- mode: latex; coding: utf-8 -*-

\chapter{Appendix}
\label{chap:appendix}

% -*- mode: latex; coding: utf-8 -*-

\section{Implementation}
\label{sec:appendix-implementation}

To start the implementation of a project, it is necessary to first think about the
goal that one wants to reach and about some basic structures and guidelines
which lead to the fulfillment of that goal.

The main goal is to have a visual animation system, which allows the creation
and rendering of visually appealing scenes, using a graphical user interface for
creation and a ray tracing based algorithm for rendering.

The thoughts to reach this goal were already developed
in~\autoref{chap:procedure}, \enquote{\nameref{chap:procedure}}, and will
therefore not be repeated again.

As stated in~\autoref{chap:procedure}, literate programming is used to implement
the components. To maintain readability only relevant code fragments are shown
in place. The whole code fragments, which are needed for tangling, are found
at~\autoref{sec:code-fragments}.

First, the implementation of the editor component is described, as it is the
basis for the whole project and also contains many concepts, that are re-used by
the player component. Before starting with the implementation it is necessary to
define requirements and some kind of framework for the implementation.

\subsection{Requirements}
\label{subsec:appendix-requirements}

At the current point of time, the requirements for running the components are
the following:

\begin{itemize}
\item A Unix derivative as operating system (Linux, macOS).
\item Python~\footnote{\url{http://www.python.org}} version 3.5.x or above
\item PyQt5~\footnote{\url{https://riverbankcomputing.com/software/pyqt/intro}}
      version 5.7 or above
\end{itemize}
\todo[inline]{Add more requirements? E.g. OpenGL?}

\subsection{Name spaces and project structure}
\label{subsec:appendix-name-spaces}

To give the whole project a structure and for being able to stick to the
thoughts established in~\autoref{chap:procedure}, it may be wise to structure
the project in analogous way as defined in~\autoref{chap:procedure}.

Therefore the whole source code shall be placed in the \textit{src} directory
underneath the main directory. The creation of the single directories is not
explicitly shown, it is done by parts of this documentation which are tangled
but not exported.

When dealing with directories and files, Python uses the term \textit{package} for
(sub-) directories and \textit{module} for files within
directories.\footnote{https://docs.python.org/3/reference/import.html\#packages}

To prevent having multiple modules having the same name, name spaces are
used.\footnote{https://docs.python.org/3/tutorial/classes.html\#python-scopes-and-namespaces}
The main name space shall be analogous to the project's name: \textit{qde}. Underneath
the source code folder \textit{src}, each sub-folder represents a package and acts
therefore also as a name space.

To actually allow a whole package and its modules being imported \textit{as modules},
it needs to have at least a file inside, called~\textit{\_\_init\_\_.py}. Those files may be
empty or they may contain regular source code such as classes or methods.

\subsection{Coding style}
\label{subsec:appendix-implementation-coding-style}

To stay consistent throughout the implementation of components, a coding style
is applied which is defined as follows.

\begin{itemize}
\item Classes use camel case, e.g. \verb+class SomeClassName+.
\item Folders respectively name-spaces use only small letters, e.g.
  \textit{foo/bar/baz}.
\item Methods are all small caps and use underscores as spaces, e.g. \verb+some_method_name+.
\item Signals are methods, which are prefixed by the word \enquote{do}, e.g. \verb+do_something+.
\item Slots are methods, which are prefixed by the word \enquote{on}, e.g. \verb+on_something+.
\item Importing is done by the \verb+from Foo import Bar+ syntax, whereas
  \verb+Foo+ is a module and \verb+Bar+ is either a module, a class or a method.
\end{itemize}

\subsubsection{Importing of modules}
\label{ssubsec:appendix-implementation-coding-style-imports}

As mentioned at~\autoref{subsec:appendix-requirements}, Python is used. Python
has~\enquote{batteries included}, which means that it offers a lot of
functionality through various modules, which have to be imported first before
using them. The same applies of course for self written modules.

Python offers multiple possibilities concerning imports, for details
see~\url{https://docs.python.org/3/tutorial/modules.html}.
\todo[inline]{Is direct url reference ok or does this need to be citation?}

However, PEP number 8 recommends to either import modules directly or to import
the needed functionality
directly.~\footnote{\url{https://www.python.org/dev/peps/pep-0020/}}. As defined
by the coding style,~\autoref{subsec:appendix-implementation-coding-style},
imports are done by the \verb+from Foo import Bar+ syntax.

The imported modules are always split up: first the system modules are imported,
modules which are provided by Python itself or by external libraries, then
project-related modules are imported.

\subsubsection{Framework for implementation}
\label{ssubsec:appendix-implementation-framework}

For also staying consistent when implementing classes and methods, it make sense
to define a rough framework for implementation, which is as follows:

\begin{itemize}
\item Define necessary signals.
\item Within the constructor,
  \begin{itemize}
    \item Set up the user interface when it is a class concerning the graphical user interface.
    \item Set up class-specific aspects, such as the name, the tile or an icon.
    \item Set up other components, used by that class.
    \item Initialize the connections, meaning hooking up the defined signals with
      corresponding methods.
  \end{itemize}
\item Implement the remaining functionality in terms of methods and slots.
\end{itemize}

Now, having defined the requirements, a project structure, a coding style and a
framework for the actual implementation, the implementation of the editor may
begin.

% -*- mode: latex; coding: utf-8 -*-

\subsection{Editor}
\label{subsec:editor}

Before diving right into the implementation of the editor, it may be good to
reconsider what shall actually be implemented, therefore what the main
functionality of the editor is and what its components are.

The quintessence of the editor application is to output a structure, be it in
the JSON format or even in bytecode, which defines an animation.

An animation is simply a composition of scenes which run in a sequential order
within a time span. A scene is at the end of its evaluation nothing else as
shader specific code which gets executed on the GPU.

To achieve this overall goal while providing the user a user-friendly
experience, several components are needed. These are the following, being
defined in~\citetitle[pp. 29 ff.]{osterwalder_qde_2016}

\begin{itemize}
\item A scene graph, allowing the creation and deletion of scenes. The scene graph
      has at least a root scene.
\item A node-based graph structure, allowing the composition of scenes using nodes
      and connections between the nodes. There exists at least a root node at
      the root scene of the scene graph.
\item A parameter window, showing parameters of the currently selected graph node.
\item A rendering window, rendering the currently selected node or scene.
\item A sequencer, allowing a time-based scheduling of defined scenes.
\end{itemize}

However, the above list is not quite complete. It is somehow intuitively clear,
that there needs to be some main component, which holds all the mentioned
components and allows a proper handling of the application. As the whole
architecture uses layers and the MVC principle (see~\autoref{subsec:code}
and~\autoref{ssubsec:mvc}), the main component is composed of a view and a
controller. A model is (at least at this point) not necessary. The view
component shall be called~\textit{main window} and its controller shall be
called~\textit{main application}.

Before implementing any of these components, the editor application needs an
entry point, that is a point where the application starts when being called.

Python does this by evaluating a special variable called~\verb+__name__+. This
value is set to \verb+'__main__'+ if the module is~\enquote{read from standard
input, a script, or from an interactive
prompt.}~\footnote{\url{https://docs.python.org/3/library/__main__.html}}

All that the entry point needs to do in case of the editor application, is
spawning the editor application, execute it and exit again, as can be seen below.

\begin{flushleft} \small
\begin{minipage}{\linewidth}\label{scrap1}\raggedright\small
\NWtarget{nuweb14}{} $\langle\,${\itshape Main entry point}\nobreak\ {\footnotesize {14}}$\,\rangle\equiv$
\vspace{-1ex}
\begin{list}{}{} \item
\mbox{}\lstinline@@\\
\mbox{}\lstinline@if __name__ == "__main__":@\\
\mbox{}\lstinline@  app = application.Application(sys.argv)@\\
\mbox{}\lstinline@  status = app.exec()@\\
\mbox{}\lstinline@  sys.exit(status)@\\
\mbox{}\lstinline@@{\NWsep}
\end{list}
\vspace{-1.5ex}
\footnotesize
\begin{list}{}{\setlength{\itemsep}{-\parsep}\setlength{\itemindent}{-\leftmargin}}
\item \NWtxtMacroRefIn\ \NWlink{nuweb21a}{21a}.

\item{}
\end{list}
\end{minipage}\vspace{4ex}
\end{flushleft}
But where to place this entry point? A very direct approach would be to
implement that main entry point within the main application controller. But when
running the editor application by calling it from the command line, calling a
controller directly may rather be confusing. Instead it is more intuitive to
have only a minimal entry point which is clearly visible as such.

Although a main entry point is defined by now, the editor application cannot be
started as there is no such thing as an editor application yet.


As stated in the requirements, see~\autoref{subsec:requirements}, Qt version 5
is used through the PyQt5 wrapper. Therefore all functionality of Qt 5 may be
used. Qt already offers a main application class, which can be used as a
controller. The class is called~\verb+QApplication+.

But what does such a main application class actually do? What is its
functionality? Very roughly sketched, such a type of application initializes
resources, enters a main loop where it stays until told to shut down. At the end
it frees resources again.

As the main application initializes resources, it act as central node between the
various layers of the architecture, initializing them and connecting them using
signals.\cite[pp. 37 --- 38]{osterwalder_qde_2016}

Due to the usage of \verb+QApplication+ as super class it is not necessary to
implement a main (event-) loop, as such is provided by Qt
itself~\footnote{http://doc.qt.io/Qt-5/qapplication.html\#exec}.

As stated above, the main application acts as entry point and as a central node
between the various layers. Therefore it needs to do at least three things:
initialize itself, set up components and connect components. This all happens
when the main application is being initialized.

\begin{flushleft} \small
\begin{minipage}{\linewidth}\label{scrap2}\raggedright\small
\NWtarget{nuweb15a}{} $\langle\,${\itshape Main application declarations}\nobreak\ {\footnotesize {15a}}$\,\rangle\equiv$
\vspace{-1ex}
\begin{list}{}{} \item
\mbox{}\lstinline@@\\
\mbox{}\lstinline@class Application(QtWidgets.QApplication):@\\
\mbox{}\lstinline@    """Main application for QDE."""@\\
\mbox{}\lstinline@@\\
\mbox{}\lstinline@    @\hbox{$\langle\,${\itshape Main application methods}\nobreak\ {\footnotesize \NWlink{nuweb15b}{15b}}$\,\rangle$}\lstinline@@\\
\mbox{}\lstinline@@{\NWsep}
\end{list}
\vspace{-1.5ex}
\footnotesize
\begin{list}{}{\setlength{\itemsep}{-\parsep}\setlength{\itemindent}{-\leftmargin}}
\item \NWtxtMacroRefIn\ \NWlink{nuweb21b}{21b}.

\item{}
\end{list}
\end{minipage}\vspace{4ex}
\end{flushleft}
\begin{flushleft} \small
\begin{minipage}{\linewidth}\label{scrap3}\raggedright\small
\NWtarget{nuweb15b}{} $\langle\,${\itshape Main application methods}\nobreak\ {\footnotesize {15b}}$\,\rangle\equiv$
\vspace{-1ex}
\begin{list}{}{} \item
\mbox{}\lstinline@@\\
\mbox{}\lstinline@@\hbox{$\langle\,${\itshape Main application constructor}\nobreak\ {\footnotesize \NWlink{nuweb15c}{15c}, \ldots\ }$\,\rangle$}\lstinline@@\\
\mbox{}\lstinline@@{\NWsep}
\end{list}
\vspace{-1.5ex}
\footnotesize
\begin{list}{}{\setlength{\itemsep}{-\parsep}\setlength{\itemindent}{-\leftmargin}}
\item \NWtxtMacroRefIn\ \NWlink{nuweb15a}{15a}.

\item{}
\end{list}
\end{minipage}\vspace{4ex}
\end{flushleft}
\begin{flushleft} \small
\begin{minipage}{\linewidth}\label{scrap4}\raggedright\small
\NWtarget{nuweb15c}{} $\langle\,${\itshape Main application constructor}\nobreak\ {\footnotesize {15c}}$\,\rangle\equiv$
\vspace{-1ex}
\begin{list}{}{} \item
\mbox{}\lstinline@@\\
\mbox{}\lstinline@def __init__(self, arguments):@\\
\mbox{}\lstinline@    """Constructor.@\\
\mbox{}\lstinline@@\\
\mbox{}\lstinline@    :param arguments: a (variable) list of arguments, that are@\\
\mbox{}\lstinline@                      passed when calling this class.@\\
\mbox{}\lstinline@    :type  argv:      list@\\
\mbox{}\lstinline@    """@\\
\mbox{}\lstinline@@\\
\mbox{}\lstinline@    @\hbox{$\langle\,${\itshape Set up internals for main application}\nobreak\ {\footnotesize \NWlink{nuweb16a}{16a}}$\,\rangle$}\lstinline@@\\
\mbox{}\lstinline@    @\hbox{$\langle\,${\itshape Set up components for main application}\nobreak\ {\footnotesize \NWlink{nuweb17d}{17d}}$\,\rangle$}\lstinline@@\\
\mbox{}\lstinline@    @\hbox{$\langle\,${\itshape Connect components for main application}\nobreak\ {\footnotesize ?}$\,\rangle$}\lstinline@@\\
\mbox{}\lstinline@@{\NWsep}
\end{list}
\vspace{-1.5ex}
\footnotesize
\begin{list}{}{\setlength{\itemsep}{-\parsep}\setlength{\itemindent}{-\leftmargin}}
\item \NWtxtMacroDefBy\ \NWlink{nuweb15c}{15c}\NWlink{nuweb18c}{, 18c}.
\item \NWtxtMacroRefIn\ \NWlink{nuweb15b}{15b}.

\item{}
\end{list}
\end{minipage}\vspace{4ex}
\end{flushleft}
Setting up the internals is straight forward: Passing any given arguments
directly to~\verb+QApplication+, setting an application icon, a name as well as
a display name.

\begin{flushleft} \small
\begin{minipage}{\linewidth}\label{scrap5}\raggedright\small
\NWtarget{nuweb16a}{} $\langle\,${\itshape Set up internals for main application}\nobreak\ {\footnotesize {16a}}$\,\rangle\equiv$
\vspace{-1ex}
\begin{list}{}{} \item
\mbox{}\lstinline@@\\
\mbox{}\lstinline@super(Application, self).__init__(arguments)@\\
\mbox{}\lstinline@self.setWindowIcon(QtGui.QIcon("assets/icons/im.png"))@\\
\mbox{}\lstinline@self.setApplicationName("QDE")@\\
\mbox{}\lstinline@self.setApplicationDisplayName("QDE")@\\
\mbox{}\lstinline@@{\NWsep}
\end{list}
\vspace{-1.5ex}
\footnotesize
\begin{list}{}{\setlength{\itemsep}{-\parsep}\setlength{\itemindent}{-\leftmargin}}
\item \NWtxtMacroRefIn\ \NWlink{nuweb15c}{15c}.

\item{}
\end{list}
\end{minipage}\vspace{4ex}
\end{flushleft}
Having the main application as a very basic implementation, the view component
of the main application, the main window, can now be implemented and then be set
up by the main application.

The main functionality of the main window is to set up the actual user
interface, containing all the views of the components. Qt offers the class
\verb+QMainWindow+ from which \verb=MainWindow= may inherit.

\begin{flushleft} \small
\begin{minipage}{\linewidth}\label{scrap6}\raggedright\small
\NWtarget{nuweb16b}{} $\langle\,${\itshape Main window declarations}\nobreak\ {\footnotesize {16b}}$\,\rangle\equiv$
\vspace{-1ex}
\begin{list}{}{} \item
\mbox{}\lstinline@@\\
\mbox{}\lstinline@class MainWindow(QtWidgets.QMainWindow):@\\
\mbox{}\lstinline@    """The main window class.@\\
\mbox{}\lstinline@    Acts as main view for the QDE editor application.@\\
\mbox{}\lstinline@    """@\\
\mbox{}\lstinline@@\\
\mbox{}\lstinline@    @\hbox{$\langle\,${\itshape Main window signals}\nobreak\ {\footnotesize \NWlink{nuweb16c}{16c}}$\,\rangle$}\lstinline@@\\
\mbox{}\lstinline@@\\
\mbox{}\lstinline@    @\hbox{$\langle\,${\itshape Main window methods}\nobreak\ {\footnotesize \NWlink{nuweb17a}{17a}, \ldots\ }$\,\rangle$}\lstinline@@\\
\mbox{}\lstinline@@\\
\mbox{}\lstinline@    @\hbox{$\langle\,${\itshape Main window slots}\nobreak\ {\footnotesize ?}$\,\rangle$}\lstinline@@\\
\mbox{}\lstinline@@{\NWsep}
\end{list}
\vspace{-1.5ex}
\footnotesize
\begin{list}{}{\setlength{\itemsep}{-\parsep}\setlength{\itemindent}{-\leftmargin}}
\item \NWtxtMacroRefIn\ \NWlink{nuweb21c}{21c}.

\item{}
\end{list}
\end{minipage}\vspace{4ex}
\end{flushleft}
For being able to shut down the application the main application and therefore
the main window need to react to a request for shutting down, either by a
keyboard shortcut or a menu command. However, \verb=MainWindow= is not able to
force \verb=Application= to quit by itself. It would be possible to pass
\verb=MainWindow= a reference to \verb=Application= but that would lead to tight
coupling and is therefore not considered as an option. Signals and slots allow
exactly such cross-layer communication without coupling components tightly.

To avoid tight coupling a signal within the main window is introduced, which
tells the main application to shut down. A fitting name for the signal might be
\verb=do_close=.

\begin{flushleft} \small
\begin{minipage}{\linewidth}\label{scrap7}\raggedright\small
\NWtarget{nuweb16c}{} $\langle\,${\itshape Main window signals}\nobreak\ {\footnotesize {16c}}$\,\rangle\equiv$
\vspace{-1ex}
\begin{list}{}{} \item
\mbox{}\lstinline@@\\
\mbox{}\lstinline@do_close = QtCore.pyqtSignal()@\\
\mbox{}\lstinline@@{\NWsep}
\end{list}
\vspace{-1.5ex}
\footnotesize
\begin{list}{}{\setlength{\itemsep}{-\parsep}\setlength{\itemindent}{-\leftmargin}}
\item \NWtxtMacroRefIn\ \NWlink{nuweb16b}{16b}.

\item{}
\end{list}
\end{minipage}\vspace{4ex}
\end{flushleft}
Now, that the signal for closing the window and the application is defined, two
additional things need to be considered: The emission of the signal by
\verb=MainWindow= itself as well as the consumption of the signal by a slot of
other classes.

The signal shall be emitted when the escape key on the keyboard is pressed or
when the corresponding menu item was selected. As there is no menu at the
moment, only the key pressed event is implemented by now.

\begin{flushleft} \small
\begin{minipage}{\linewidth}\label{scrap8}\raggedright\small
\NWtarget{nuweb17a}{} $\langle\,${\itshape Main window methods}\nobreak\ {\footnotesize {17a}}$\,\rangle\equiv$
\vspace{-1ex}
\begin{list}{}{} \item
\mbox{}\lstinline@@\\
\mbox{}\lstinline@@\hbox{$\langle\,${\itshape Main window constructor}\nobreak\ {\footnotesize \NWlink{nuweb17b}{17b}, \ldots\ }$\,\rangle$}\lstinline@@\\
\mbox{}\lstinline@@\hbox{$\langle\,${\itshape Main window key press event}\nobreak\ {\footnotesize \NWlink{nuweb17c}{17c}}$\,\rangle$}\lstinline@@\\
\mbox{}\lstinline@@{\NWsep}
\end{list}
\vspace{-1.5ex}
\footnotesize
\begin{list}{}{\setlength{\itemsep}{-\parsep}\setlength{\itemindent}{-\leftmargin}}
\item \NWtxtMacroDefBy\ \NWlink{nuweb17a}{17a}\NWlink{nuweb18a}{, 18a}.
\item \NWtxtMacroRefIn\ \NWlink{nuweb16b}{16b}.

\item{}
\end{list}
\end{minipage}\vspace{4ex}
\end{flushleft}
\begin{flushleft} \small
\begin{minipage}{\linewidth}\label{scrap9}\raggedright\small
\NWtarget{nuweb17b}{} $\langle\,${\itshape Main window constructor}\nobreak\ {\footnotesize {17b}}$\,\rangle\equiv$
\vspace{-1ex}
\begin{list}{}{} \item
\mbox{}\lstinline@@\\
\mbox{}\lstinline@def __init__(self):@\\
\mbox{}\lstinline@    """Constructor."""@\\
\mbox{}\lstinline@@\\
\mbox{}\lstinline@    super(MainWindow, self).__init__()@\\
\mbox{}\lstinline@@{\NWsep}
\end{list}
\vspace{-1.5ex}
\footnotesize
\begin{list}{}{\setlength{\itemsep}{-\parsep}\setlength{\itemindent}{-\leftmargin}}
\item \NWtxtMacroDefBy\ \NWlink{nuweb17b}{17b}\NWlink{nuweb18b}{, 18b}.
\item \NWtxtMacroRefIn\ \NWlink{nuweb17a}{17a}.

\item{}
\end{list}
\end{minipage}\vspace{4ex}
\end{flushleft}
\begin{flushleft} \small
\begin{minipage}{\linewidth}\label{scrap10}\raggedright\small
\NWtarget{nuweb17c}{} $\langle\,${\itshape Main window key press event}\nobreak\ {\footnotesize {17c}}$\,\rangle\equiv$
\vspace{-1ex}
\begin{list}{}{} \item
\mbox{}\lstinline@@\\
\mbox{}\lstinline@def keyPressEvent(self, event):@\\
\mbox{}\lstinline@    """Gets triggered when a key press event is raised.@\\
\mbox{}\lstinline@@\\
\mbox{}\lstinline@    :param event: holds the triggered event.@\\
\mbox{}\lstinline@    :type  event: QKeyEvent@\\
\mbox{}\lstinline@    """@\\
\mbox{}\lstinline@@\\
\mbox{}\lstinline@    if event.key() == QtCore.Qt.Key_Escape:@\\
\mbox{}\lstinline@        self.do_close.emit()@\\
\mbox{}\lstinline@    else:@\\
\mbox{}\lstinline@        super(MainWindow, self).keyPressEvent(event)@\\
\mbox{}\lstinline@@\\
\mbox{}\lstinline@@{\NWsep}
\end{list}
\vspace{-1.5ex}
\footnotesize
\begin{list}{}{\setlength{\itemsep}{-\parsep}\setlength{\itemindent}{-\leftmargin}}
\item \NWtxtMacroRefIn\ \NWlink{nuweb17a}{17a}.

\item{}
\end{list}
\end{minipage}\vspace{4ex}
\end{flushleft}
% For emitting the signal when selecting a menu entry, an action needs to be
% defined which is then attached to the menu entry. The action emits a signal as
% soon as the menu entry was clicked. It is however not possible to trigger the
% defined \verb+do_close+ signal using the actions signal. There a slot needs to
% be defined which then in its turn triggers \verb+do_close+.

The main window can now be set up by the main application controller, which also
listens to the \verb=do_close= signal through the inherited \verb=quit= slot.

\begin{flushleft} \small
\begin{minipage}{\linewidth}\label{scrap11}\raggedright\small
\NWtarget{nuweb17d}{} $\langle\,${\itshape Set up components for main application}\nobreak\ {\footnotesize {17d}}$\,\rangle\equiv$
\vspace{-1ex}
\begin{list}{}{} \item
\mbox{}\lstinline@@\\
\mbox{}\lstinline@self.main_window = qde_main_window.MainWindow()@\\
\mbox{}\lstinline@self.main_window.do_close.connect(self.quit)@\\
\mbox{}\lstinline@@{\NWsep}
\end{list}
\vspace{-1.5ex}
\footnotesize
\begin{list}{}{\setlength{\itemsep}{-\parsep}\setlength{\itemindent}{-\leftmargin}}
\item \NWtxtMacroRefIn\ \NWlink{nuweb15c}{15c}.

\item{}
\end{list}
\end{minipage}\vspace{4ex}
\end{flushleft}
The used view component for the main window, \verb+QMainWindow+, needs at least
a central widget with a layout for being
rendered.~\footnote{http://doc.qt.io/qt-5/qmainwindow.html\#creating-main-window-components}

As the main window will set up and hold the whole layout for the application
through multiple view components, a method \verb+setup_ui+ is introduced, which
sets up the whole layout.

\begin{flushleft} \small
\begin{minipage}{\linewidth}\label{scrap12}\raggedright\small
\NWtarget{nuweb18a}{} $\langle\,${\itshape Main window methods}\nobreak\ {\footnotesize {18a}}$\,\rangle+\equiv$
\vspace{-1ex}
\begin{list}{}{} \item
\mbox{}\lstinline@@\\
\mbox{}\lstinline@def setup_ui(self):@\\
\mbox{}\lstinline@    """Sets up the user interface specific components."""@\\
\mbox{}\lstinline@@\\
\mbox{}\lstinline@    self.setObjectName('MainWindow')@\\
\mbox{}\lstinline@    self.setWindowTitle('QDE')@\\
\mbox{}\lstinline@    self.resize(1024, 768)@\\
\mbox{}\lstinline@    self.move(100, 100)@\\
\mbox{}\lstinline@    # Ensure that the window is not hidden behind other windows@\\
\mbox{}\lstinline@    self.activateWindow()@\\
\mbox{}\lstinline@@\\
\mbox{}\lstinline@    central_widget = QtWidgets.QWidget(self)@\\
\mbox{}\lstinline@    central_widget.setObjectName('central_widget')@\\
\mbox{}\lstinline@    grid_layout = QtWidgets.QGridLayout(central_widget)@\\
\mbox{}\lstinline@    central_widget.setLayout(grid_layout)@\\
\mbox{}\lstinline@    self.setCentralWidget(central_widget)@\\
\mbox{}\lstinline@    self.statusBar().showMessage('Ready.')@\\
\mbox{}\lstinline@@{\NWsep}
\end{list}
\vspace{-1.5ex}
\footnotesize
\begin{list}{}{\setlength{\itemsep}{-\parsep}\setlength{\itemindent}{-\leftmargin}}
\item \NWtxtMacroDefBy\ \NWlink{nuweb17a}{17a}\NWlink{nuweb18a}{, 18a}.
\item \NWtxtMacroRefIn\ \NWlink{nuweb16b}{16b}.

\item{}
\end{list}
\end{minipage}\vspace{4ex}
\end{flushleft}
\begin{flushleft} \small
\begin{minipage}{\linewidth}\label{scrap13}\raggedright\small
\NWtarget{nuweb18b}{} $\langle\,${\itshape Main window constructor}\nobreak\ {\footnotesize {18b}}$\,\rangle+\equiv$
\vspace{-1ex}
\begin{list}{}{} \item
\mbox{}\lstinline@@\\
\mbox{}\lstinline@self.setup_ui()@\\
\mbox{}\lstinline@@{\NWsep}
\end{list}
\vspace{-1.5ex}
\footnotesize
\begin{list}{}{\setlength{\itemsep}{-\parsep}\setlength{\itemindent}{-\leftmargin}}
\item \NWtxtMacroDefBy\ \NWlink{nuweb17b}{17b}\NWlink{nuweb18b}{, 18b}.
\item \NWtxtMacroRefIn\ \NWlink{nuweb17a}{17a}.

\item{}
\end{list}
\end{minipage}\vspace{4ex}
\end{flushleft}
The main window can now be shown by the main application controller.

\begin{flushleft} \small
\begin{minipage}{\linewidth}\label{scrap14}\raggedright\small
\NWtarget{nuweb18c}{} $\langle\,${\itshape Main application constructor}\nobreak\ {\footnotesize {18c}}$\,\rangle+\equiv$
\vspace{-1ex}
\begin{list}{}{} \item
\mbox{}\lstinline@@\\
\mbox{}\lstinline@@\\
\mbox{}\lstinline@self.main_window.show()@\\
\mbox{}\lstinline@@{\NWsep}
\end{list}
\vspace{-1.5ex}
\footnotesize
\begin{list}{}{\setlength{\itemsep}{-\parsep}\setlength{\itemindent}{-\leftmargin}}
\item \NWtxtMacroDefBy\ \NWlink{nuweb15c}{15c}\NWlink{nuweb18c}{, 18c}.
\item \NWtxtMacroRefIn\ \NWlink{nuweb15b}{15b}.

\item{}
\end{list}
\end{minipage}\vspace{4ex}
\end{flushleft}
% -*- mode: latex; coding: utf-8 -*-

\section{Work log}
\label{sec:work-log}

\begin{loggentry}{2017-02-20}{Mon}
  Set up and structure the document initially.
\end{loggentry}

\begin{loggentry}{2017-02-21}{Tue}
  Re-structure the document, add first contents of the implementation. Add first
  tries to tangle the code. he document initially.
\end{loggentry}

\begin{loggentry}{2017-02-22}{Wed}
  Provide further content concerning the implementation: Introduce
  name-spaces/initializers, first steps for a logging facility.
\end{loggentry}

\begin{loggentry}{2017-02-23}{Thu}
  Extend logging facility, provide (unit-) tests. Restructure the documentation.
\end{loggentry}

\begin{loggentry}{2017-02-24}{Fri}
  Adapt document to output LaTeX code as desired, change styling. Begin
development of the applications' main routine.
\end{loggentry}

\begin{loggentry}{2017-02-27}{Mon}
  Remove (unit-) tests from main document and put them into appendix instead.
  Begin explaining literate programming.
\end{loggentry}

\begin{loggentry}{2017-02-28}{Tue}
  Provide a first draft for objectives and limitations.
  Re-structure the document. Correct LaTeX output.
\end{loggentry}

\begin{loggentry}{2017-03-01}{Wed}
  Remove split files, re-add everything to index, add
  objectives.
\end{loggentry}

\begin{loggentry}{2017-03-02}{Thu}
  Set up project schedule. Tangle everything instead of
  doing things manually. Begin changing language to English instead of German.
  Re-add make targets for cleaning and building the source code.
\end{loggentry}

\begin{loggentry}{2017-03-03}{Fri}
  Keep work log up to date. Revise and finish chapter about
  name-spaces and the project structure for now.
\end{loggentry}

\begin{loggentry}{2017-03-04}{Sat}
  Finish translating all already written texts from German
  to English. Describe the main entry point of the application as well as the
  main application itself.
\end{loggentry}

\begin{loggentry}{2017-03-05}{Sun}
  Finish chapter about the main entry point and the main
  application for now, start describing the main window and implement its
  functionality. Keep the work log up to date. Fiddle with references and
  LaTeX export. Find a bug: main\_window needs to be attached to a class, by
  using the \textit{self} keyword, otherwise the window does not get shown.
  Introduce new make targets: one to clean Python cache files (*.pyc) and one
  to run the editor application directly.
\end{loggentry}

\begin{loggentry}{2017-03-06}{Mon}
  Update the work log. Add an image of the editor as well as
  the project schedule. Add the implementation of the main window's layout.
  Implement the scene domain model. Move keyPressEvent to its own source
  block instead of expanding the methods of the main window directly. Add a
  section about (the architecture's) layers to the principles section. Add
  Dr. Eric Dubuis as an expert to the involved persons. Introduce the 'verb'
  macro for having nicer verbatim blocks. Use the given image-width for
  inline images in org-mode when available.
\end{loggentry}

\begin{loggentry}{2017-03-07}{Tue}
  Expand the layering principles by adding a section about
  the model-view-controller pattern and introduce view models. Explain and
  implement the data- and the view model for scene graph items.
\end{loggentry}

\begin{loggentry}{2017-03-08}{Wed}
  Implement the controller for handling the scene graph.
  Allow the semi-automatic creation of an API documentation by introducing
  Sphinx. Introduce new make targets for creating the API documentation as
  RST and as HTML.
\end{loggentry}

\begin{loggentry}{2017-03-10}{Fri}
  Implement the scene graph view as widget and integrate it
  into the application. Update the work log. Fix typing errors. Start to
  implement missing methods in the scene graph controller for being able to
  use the scene graph widget.
\end{loggentry}

\begin{loggentry}{2017-03-13}{Mon}
  Implement the scene view model. Initialize such a model
  within the scene graph view model. Implement the =headerData= as well as
  the =data= methods of the scene graph controller. Update the work log. Add
  an image of the editor's current state. Continue implementation of the
  scene graph view model.
\end{loggentry}

\begin{loggentry}{2017-03-14}{Tue}
  Continue the implementation of the scene graph view model.
  Implement logging. Implement logging. Implement logging. Implement logging
  functionality. Log whenever a node is added or removed from the scene graph
  view.
\end{loggentry}

\begin{loggentry}{2017-03-15}{Wed}
  Move logging further down in structure. Add connections
  between scene graph view and controller. Finish implementing the adding and
  removal of scene graph items. Update the work log.

  Next steps: (Re-) Introduce logging. Begin implementing the node graph.
\end{loggentry}

\begin{loggentry}{2017-03-16}{Thu}
  Run sphinx apidoc when creating the HTML documentation.
  Add an illustration about the state of the editor after finishing the
  implementation of the scene graph. Change width of the images to be 50% of
  the text width. Name slots of the scene graph view explicitly to maintain
  sanity. Re-add logging chapter with a corresponding introduction. Fix display
  of code listings. Keep work log up to date. Add missing TODO annotations to
  headings.

  Next steps: Continue implementing the node graph.
\end{loggentry}

\begin{loggentry}{2017-03-17}{Fri}
  Change verbatim output to be less intrusive, update to do
  tags, begin adding references do code fragment definitions, begin implement
  the node graph. Move chapters into separate org files.
\end{loggentry}

\begin{loggentry}{2017-03-20}{Mon}
  Re-think how to implement node definitions and revise
  therefore the chapter about the node graph component, fix various
  typographic errors, expand and change the Makefile, keep the work log up
  to date.
\end{loggentry}

\begin{loggentry}{2017-03-21}{Tue}
  Re-think how to implement node definitions.
\end{loggentry}

\begin{loggentry}{2017-03-22}{Wed}
  Re-think how to implement node definitions and nodes. Begin adding
  notes about how to implement nodes.
\end{loggentry}

\begin{loggentry}{2017-03-23}{Thu}
  Expand notes about the node implementation, begin writing
  the actual node implementation down, keep the work log up to date.
\end{loggentry}

\begin{loggentry}{2017-03-24}{Fri}
  Attend a meeting with Prof. Fuhrer, change and expand the
  chapter about node implementation according to the before made thoughts,
  begin implementing the node graph structure, keep the work log up to date.
\end{loggentry}
\section{Requirements}
\label{sec:requirements}
% 
% This chapter describes the requirements to extract the source code out of this
% documentation using /tangling/.
% 
% At the current point of time, the requirements are the following:
% 
% - A Unix derivative as operating system (Linux, macOS).
% - Python version 3.5.x or up[fn:3:https://www.python.org].
% - Pyenv[fn:4:https://github.com/yyuu/pyenv].
% - Pyenv-virtualenv[fn:5:https://github.com/yyuu/pyenv-virtualenv].
% 
% The first step is to install a matching version of python for the usage within
% the virtual environment. The available Python versions may be listed as follows.
% 
% #+ATTR_LaTeX: :options fontsize=\footnotesize,linenos,bgcolor=bashcodebg
% #+CAPTION:    Listing all available versions of Python for use in Pyenv.
% #+NAME:       fig:impl-pyenv-list
% #+BEGIN_SRC bash
% pyenv install --list
% #+END_SRC
% 
% The desired version may be installed as follows. This example shows the
% installation of version 3.6.0.
% 
% #+ATTR_LaTeX: :options fontsize=\footnotesize,linenos,bgcolor=bashcodebg
% #+CAPTION:    Installation of Python version 3.6.0 for the usage with Pyenv.
% #+NAME:       fig:impl-pyenv-install
% #+BEGIN_SRC bash
% install 3.6.0
% #+END_SRC
% 
% It is highly recommended to create and use a project-specific virtual Python
% environment. All packages, that are required for this project are installed
% within this virtual environment protecting the operating systems' Python
% packages.
% First the desired version of Python has to be specified, then the desired name
% of the virtual environment.
% 
% #+ATTR_LaTeX: :options fontsize=\footnotesize,linenos,bgcolor=bashcodebg
% #+CAPTION:    Creation of the virtual environment =qde= for Python using version 3.6.0 of Python.
% #+NAME:       fig:impl-pyenv-venv
% #+BEGIN_SRC bash
% pyenv virtualenv 3.6.0 qde
% #+END_SRC
% 
% All required dependencies for the project may now safely be installed. Those are
% listed in the file =python_requirements.txt= and are installed using =pip=.
% 
% #+ATTR_LaTeX: :options fontsize=\footnotesize,linenos,bgcolor=bashcodebg
% #+CAPTION:    Installation of the projects' required dependencies.
% #+NAME:       fig:impl-pip-install
% #+BEGIN_SRC bash
% pip install -r python_requirements.txt
% #+END_SRC
% 
% All requirements and dependencies are now met and the actual implementation of
% the project may begin now.

\section{Directory structure and name-spaces}
\label{sec:directory-structure}

This chapter describes the planned directory structure as well as how the usage
of name-spaces is intended.

% The whole source code shall be placed in the =src= directory underneath the main
% directory. The creation of the single directories is not explicitly shown
% respectively done, instead the =:mkdirp= option provided by the source code
% block structure is used[fn:11:http://orgmode.org/manual/mkdirp.html#mkdirp]. The
% option has the same effect as would have =mkdir -p [directory/subdirectory]=: It
% creates all needed (sub-) directories, even when tangling a file. This prevents
% the tedious and non-interesting creation of directories within this document.
% 
% When dealing with directories and files, Python uses the term /package/ for a
% (sub-) directories and /module/ for files within directories, that is
% modules.[fn:13:https://docs.python.org/3/reference/import.html#packages]
% 
% To prevent having multiple modules having the same name, name-spaces are
% used[fn:6:https://docs.python.org/3/tutorial/classes.html#python-scopes-and-namespaces].
% The main name-space shall be analogous to the projects' name: =qde=. Underneath
% the source code folder =src=, each sub-folder represents a package and acts
% therefore also as a name-space.
% 
% To actually allow a whole package and its modules being imported /as modules/,
% it needs to have at least a file inside called =__init__.py=. Those files may be
% empty or they may contain regular source code such as classes or methods.
% 
% The first stage of the project shows the creation of the /editor/ component, as
% it provides the possibility of creating and editing real-time animations which
% may then be played back by the /player/ component\cite[p. 29]{osterwalder_qde_2016}.

\section{Code fragments}
\label{sec:code-fragments}

\begin{flushleft} \small
\begin{minipage}{\linewidth}\label{scrap15}\raggedright\small
\NWtarget{nuweb21a}{} \verb@"../src/editor.py"@\nobreak\ {\footnotesize {21a}}$\equiv$
\vspace{-1ex}
\begin{list}{}{} \item
\mbox{}\lstinline@#!/usr/bin/python@\\
\mbox{}\lstinline@# -*- coding: utf-8 -*-@\\
\mbox{}\lstinline@@\\
\mbox{}\lstinline@""" Main entry point for the QDE editor application. """@\\
\mbox{}\lstinline@@\\
\mbox{}\lstinline@# System imports@\\
\mbox{}\lstinline@import sys@\\
\mbox{}\lstinline@@\\
\mbox{}\lstinline@# Project imports@\\
\mbox{}\lstinline@from qde.editor.application import application@\\
\mbox{}\lstinline@@\\
\mbox{}\lstinline@@\hbox{$\langle\,${\itshape Main entry point}\nobreak\ {\footnotesize \NWlink{nuweb14}{14}}$\,\rangle$}\lstinline@@\\
\mbox{}\lstinline@@{\NWsep}
\end{list}
\vspace{-1.5ex}
\footnotesize
\begin{list}{}{\setlength{\itemsep}{-\parsep}\setlength{\itemindent}{-\leftmargin}}

\item{}
\end{list}
\end{minipage}\vspace{4ex}
\end{flushleft}
\begin{flushleft} \small
\begin{minipage}{\linewidth}\label{scrap16}\raggedright\small
\NWtarget{nuweb21b}{} \verb@"../src/qde/editor/application/application.py"@\nobreak\ {\footnotesize {21b}}$\equiv$
\vspace{-1ex}
\begin{list}{}{} \item
\mbox{}\lstinline@#!/usr/bin/python@\\
\mbox{}\lstinline@# -*- coding: utf-8 -*-@\\
\mbox{}\lstinline@@\\
\mbox{}\lstinline@"""Main application module for the QDE editor."""@\\
\mbox{}\lstinline@@\\
\mbox{}\lstinline@# System imports@\\
\mbox{}\lstinline@from PyQt5 import QtGui@\\
\mbox{}\lstinline@from PyQt5 import QtWidgets@\\
\mbox{}\lstinline@@\\
\mbox{}\lstinline@# Project imports@\\
\mbox{}\lstinline@from qde.editor.gui import main_window as qde_main_window@\\
\mbox{}\lstinline@@\\
\mbox{}\lstinline@@\\
\mbox{}\lstinline@@\hbox{$\langle\,${\itshape Main application declarations}\nobreak\ {\footnotesize \NWlink{nuweb15a}{15a}}$\,\rangle$}\lstinline@@\\
\mbox{}\lstinline@@{\NWsep}
\end{list}
\vspace{-1.5ex}
\footnotesize
\begin{list}{}{\setlength{\itemsep}{-\parsep}\setlength{\itemindent}{-\leftmargin}}

\item{}
\end{list}
\end{minipage}\vspace{4ex}
\end{flushleft}
\begin{flushleft} \small
\begin{minipage}{\linewidth}\label{scrap17}\raggedright\small
\NWtarget{nuweb21c}{} \verb@"../src/qde/editor/gui/main_window.py"@\nobreak\ {\footnotesize {21c}}$\equiv$
\vspace{-1ex}
\begin{list}{}{} \item
\mbox{}\lstinline@#!/usr/bin/python@\\
\mbox{}\lstinline@# -*- coding: utf-8 -*-@\\
\mbox{}\lstinline@@\\
\mbox{}\lstinline@""" Module holding the main application window. """@\\
\mbox{}\lstinline@@\\
\mbox{}\lstinline@# System imports@\\
\mbox{}\lstinline@from PyQt5 import QtCore@\\
\mbox{}\lstinline@from PyQt5 import QtWidgets@\\
\mbox{}\lstinline@@\\
\mbox{}\lstinline@# Project imports@\\
\mbox{}\lstinline@@\\
\mbox{}\lstinline@@\\
\mbox{}\lstinline@@\hbox{$\langle\,${\itshape Main window declarations}\nobreak\ {\footnotesize \NWlink{nuweb16b}{16b}}$\,\rangle$}\lstinline@@\\
\mbox{}\lstinline@@{\NWsep}
\end{list}
\vspace{-1.5ex}
\footnotesize
\begin{list}{}{\setlength{\itemsep}{-\parsep}\setlength{\itemindent}{-\leftmargin}}

\item{}
\end{list}
\end{minipage}\vspace{4ex}
\end{flushleft}
\end{document}