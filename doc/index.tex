% Created 2017-03-02 Thu 22:37
% Intended LaTeX compiler: pdflatex
\documentclass[10pt, openright, notitlepage]{scrreprt}
\usepackage[utf8]{inputenc}
\usepackage[T1]{fontenc}
\usepackage{graphicx}
\usepackage{grffile}
\usepackage{longtable}
\usepackage{wrapfig}
\usepackage{rotating}
\usepackage[normalem]{ulem}
\usepackage{amsmath}
\usepackage{textcomp}
\usepackage{amssymb}
\usepackage{capt-of}
\usepackage{hyperref}
\usepackage{minted}
\usepackage[a4paper, left=25mm, right=25mm, top=27mm, headheight=20mm, headsep=10mm, textheight=242mm, footskip=15mm]{geometry}
\usepackage[backend=biber, style=ieee]{biblatex}
\usepackage[dvipsnames]{xcolor}
% Definition of colors
%---------------------------------------------------------------------------
\RequirePackage{color}
\definecolor{linkblue}{rgb}{0,0,0.8}       % Standard
\definecolor{darkblue}{rgb}{0,0.08,0.45}   % Dark blue
\definecolor{bfhgrey}{rgb}{0.41,0.49,0.57} % BFH grey
\definecolor{linkcolor}{rgb}{0,0,0}
\colorlet{Black}{black}
\definecolor{keywords}{rgb}{255,0,0}
\definecolor{red}{rgb}{0.6,0,0}
\definecolor{green}{rgb}{0,0.5,0}
\definecolor{blue}{rgb}{0,0,0.5}
% Syntax colors
\definecolor{syntaxRed}{rgb}{0.6,0,0}
\definecolor{syntaxBlue}{rgb}{0,0,0.5}
\definecolor{syntaxComment}{rgb}{0,0.5,0}
% Background colors
\definecolor{syntaxBackground}{rgb}{0.95, 0.95, 0.95}
%---------------------------------------------------------------------------
\usepackage{tcolorbox}
\usepackage{pgfgantt}
\usepackage{float}
\restylefloat{listing}
\tcbuselibrary{minted,skins}
\definecolor{bashcodebg}{rgb}{0.85,0.85,0.85}
\addbibresource{bibliography.bib}
\author{Sven Osterwalder\thanks{sven.osterwalder@students.bfh.ch}}
\date{February 20, 2017}
\title{QDE --- A visual animation system.\\\medskip
\large MTE-7103: Master-Thesis}
\hypersetup{
 pdfauthor={Sven Osterwalder},
 pdftitle={QDE --- A visual animation system.},
 pdfkeywords={},
 pdfsubject={},
 pdfcreator={Emacs 25.1.1 (Org mode 9.0.5)}, 
 pdflang={English}}
\begin{document}

\maketitle
\tableofcontents


\chapter{{\bfseries\sffamily TODO} Introduction}
\label{sec:orgeccb199}

[Introduction here].

\chapter{{\bfseries\sffamily TODO} Administrative aspects}
\label{sec:org2ed9020}

Some administrative aspects of this thesis are covered, while they are not
required for the understanding of the result.

The whole documentation uses the male form, whereby both genera are equally
meant.

\section{Involved persons}
\label{sec:org886fbb7}

\begin{center}
\begin{tabular}{lll}
Author & Sven Osterwalder\footnotemark & \\
Supervisor & Prof. Claude Fuhrer\footnotemark & \emph{Supervises the student doing the thesis}\\
\end{tabular}
\end{center}\footnotetext[1]{\label{orgcf0e761}sven.osterwalder@students.bfh.ch}\footnotetext[2]{\label{orge05a87b}claude.fuhrer@bfh.ch}

\section{Structure of the documentation}
\label{sec:orgcd70a95}

This thesis is structured as follows:

\begin{itemize}
\item Introduction
\item Objectives and limitations
\item Procedure
\item Implementation
\item Conclusion
\end{itemize}

\section{Deliverable results}
\label{sec:orga937497}

\begin{itemize}
\item Report
\item Implementation
\end{itemize}

\chapter{{\bfseries\sffamily TODO} Scope}
\label{sec:org47e46bc}

\section{Motivation}
\label{sec:org2e473f6}

[Motivation.]

\section{Objectives and limitations}
\label{sec:org64e5095}

[Objectives and limitations.]

\section{Preliminary activities}
\label{sec:org5d56d0d}

[Preliminary activities.]

\section{New learning contents}
\label{sec:orgd2882be}

[New learning contents.]

\chapter{{\bfseries\sffamily TODO} Procedure}
\label{sec:orgbef0d60}
\section{Organization of work}
\label{sec:orgc11a324}
\subsection{Meetings}
\label{sec:org4a38f3a}

Various meetings with the supervising professor, Mr. Claude Fuhrer, helped
reaching the defined goals and preventing erroneous directions of the thesis.
The supervisor supported the author of this thesis by providing suggestions
throughout the held meetings. The minutes of the meetings may be found under
\label{org3846b6e}.

\subsection{Phases of the project and milestones}
\label{sec:orga4ad2cc}


\begin{center}
\begin{tabular}{llr}
Phase & Description & Week / 2017\\
\hline
Start of the project &  & 8\\
Definition of objectives and limitations &  & 8-9\\
Documentation and development &  & 8-30\\
Corrections &  & 30-31\\
Preparation of the thesis' defense &  & 31-32\\
\end{tabular}
\end{center}

\begin{center}
\begin{tabular}{llr}
Milestone & Description & End of week / 2017\\
\hline
Project structure is set up &  & 8\\
Mandatory project goals are reached &  & 30\\
Hand-in of the thesis &  & 31\\
Defense of the thesis &  & 32\\
\end{tabular}
\end{center}

\subsection{Literate programming}
\label{sec:org761ae29}

This thesis' implementation is done by a procedure named ``literate
programming'', invented by Donald Knuth. What this means, is that the
documentation as well as the code for the resulting program reside in the same
file. The documentation is then \emph{weaved} into a separate document, which may be
any by the editor support format. The code of the program is \emph{tangled} into a
run-able computer program.

\begin{center}
\fbox{
\begin{minipage}[c]{.6\textwidth}
\textbf{\textsf{\textsc{TODO}}} Provide more information about literate programming.

\rule[.8em]{\textwidth}{2pt}

Citations, explain fragments, explain referencing
fragments, code structure does not have to be ``normal''
\end{minipage}
}
\end{center}

Originally it was planned to develop this thesis' application test driven,
providing (unit-) test-cases first and implementing the functionality
afterwards. Initial trails showed quickly that this method, in company with
literate programming, would exaggerate the effort needed. Therefore conventional
testing is used. Test are developed after implementing functionality and run
separately. A coverage as high as possible is intended. Test cases are \emph{tangled}
too, and may be found in the appendix.
\begin{center}
\fbox{
\begin{minipage}[c]{.6\textwidth}
\textbf{\textsf{\textsc{TODO}}} Insert reference/link to test cases here.

\end{minipage}
}
\end{center}

\section{Standards and principles}
\label{sec:orge461548}
\subsection{Code}
\label{sec:org029fcec}

[Code.]

\subsection{Diagrams}
\label{sec:orgf500ace}

[Diagrams.]

\subsection{Project structure}
\label{sec:org360f7d3}

[Project structure.]

\chapter{{\bfseries\sffamily TODO} Implementation}
\label{sec:orgcfeb7e6}

\section{Requirements}
\label{sec:org7d9cee3}

This chapter describes the requirements to extract the source code out of this
documentation using \emph{tangling}.

At the current point of time, the requirements are the following:

\begin{itemize}
\item A Unix derivative as operating system (Linux, macOS).
\item Python version 3.5.x or up\footnote{\url{https://www.python.org}}.
\item Pyenv\footnote{\url{https://github.com/yyuu/pyenv}}.
\item Pyenv-virtualenv\footnote{\url{https://github.com/yyuu/pyenv-virtualenv}}.
\end{itemize}

The first step is to install a matching version of python for the usage within
the virtual environment. The available Python versions may be listed as follows.

\begin{listing}[H]
\begin{minted}[,fontsize=\footnotesize,linenos,bgcolor=bashcodebg]{bash}
pyenv install --list
\end{minted}
\caption{\label{org579cab3}
Listing all available versions of Python for use in Pyenv.}
\end{listing}

The desired version may be installed as follows. This example shows the
installation of version 3.6.0.

\begin{listing}[H]
\begin{minted}[,fontsize=\footnotesize,linenos,bgcolor=bashcodebg]{bash}
install 3.6.0
\end{minted}
\caption{\label{org71b3761}
Installation of Python version 3.6.0 for the usage with Pyenv.}
\end{listing}

It is highly recommended to create and use a project-specific virtual Python
environment. All packages, that are required for this project are installed
within this virtual environment protecting the operating systems' Python
packages.
First the desired version of Python has to be specified, then the desired name
of the virtual environment.

\begin{listing}[H]
\begin{minted}[,fontsize=\footnotesize,linenos,bgcolor=bashcodebg]{bash}
pyenv virtualenv 3.6.0 qde
\end{minted}
\caption{\label{org645be09}
Creation of the virtual environment \texttt{qde} for Python using version 3.6.0 of Python.}
\end{listing}

All required dependencies for the project may now safely be installed. Those are
listed in the file \texttt{python\_requirements.txt} and are installed using \texttt{pip}.

\begin{listing}[H]
\begin{minted}[,fontsize=\footnotesize,linenos,bgcolor=bashcodebg]{bash}
pip install -r python_requirements.txt
\end{minted}
\caption{\label{org7e6f3c8}
Installation of the projects' required dependencies.}
\end{listing}

All requirements and dependencies are now met and the actual implementation of
the project may begin now.

\section{Project structure}
\label{sec:org06c4f56}

This chapter describes the planned directory structure as well as an instruction
on how to set up that structure.

The whole source code shall be placed in the \texttt{src} directory underneath the main
directory.

To prevent multiple modules having the same name, name spaces are
used\footnote{\url{https://docs.python.org/3/tutorial/classes.html\#python-scopes-and-namespaces}}.
The main name space shall be analogous to the projects' name: \texttt{qde}.

In der ersten Phase des Projektes soll der Editor erstellt werden. Dieser dient
der Erstellung und Verwaltung von Echtzeit-Animationen \cite[S. 29]{osterwalder_qde_2016}.

\section{Editor}
\label{sec:orgc5a3405}

Der Editor soll sich im Verzeichnis \texttt{editor} unterhalb des \texttt{src/qde}-Verzeichnisses
befinden.

Um sicherzustellen, dass Module als solche verwendet werden können, muss pro
Modul und Namespace eine Datei zur Initialisierung vorhanden sein. Es handelt sich
dabei um Dateien namens \texttt{\_\_init\_\_.py}, welche im minimalen Fall leer sind. Diese
können aber auch regulären Programmcode, wie zum Beispiel Klassen oder Methoden
enthalten.

Im weiteren Verlauf des Dokumentes wird darauf verzichtet diese Dateien explizit
zu erwähnen, sie werden direkt in den entsprechenden Codeblöcken erstellt und
als gegeben angesehen.

Nun kann mit der eigentlichen Erstellung des Editors begonnen werden.

Der Einstiegspunkt einer Qt-Applikation mit grafischer Oberfläche ist die Klasse
\texttt{QtApplication}. Gemäss \footnote{\url{http://doc.qt.io/Qt-5/qapplication.html}} kann die
Klasse direkt instanziert und benutzt werden, es ist unter Umständen jedoch
sinnvoller die Klasse zu kapseln, was schlussendlich eine höhere Flexibilität
bei der Umsetzung bietet. Es soll daher die Klasse \texttt{Application} erstellt
werden, welche diese Abstraktion bietet.

An dieser Stelle macht es Sinn, sich zu überlegen, welche Funktionalität die
Applikation selbst haben soll. Es ist nicht nötig selbst einen Event-Loop zu
implementieren, da ein solcher bereits durch Qt vorhanden
ist\footnote{\url{http://doc.qt.io/Qt-5/qapplication.html\#exec}}.

Die Applikation hat die Aufgabe die Kernelemente der Applikation zu
initialisieren. So fungiert das Modul als Knotenpunkt zwischen den
verschiedenen Ebenen der Architektur, indem es diese mittels Signalen
verbindet.\cite[S. 37 bis 38]{osterwalder_qde_2016}

Weiter soll es nützliche Schnittstellen, wie zum Beispiel das Protokollieren
von Meldungen, bereitstellen. Und schliesslich soll das Modul eine Möglichkeit
bieten beim Verlassen der Applikation zusätzliche Aufgaben, wie etwa das
Entfernen von temporären Dateien, zu bieten.

Da es sehr nützlich ist, den Zustand einer Applikation jederzeit in Form von
gezielten Ausgaben nachvollziehen zu können, bietet es sich an als ersten
Schritt ein Modul zur Protokollierung zu implementieren.
Protokollierung ist ein sehr zentrales Element, daher wird das Modul im
Namespace \texttt{foundation} erstellt.

Die (Datei-) Struktur zur Erstellung und Benennung der Module erfolgt ab diesem
Zeitpunkt nach dem Schichten-Modell gemäss \cite[S. 40]{osterwalder_qde_2016}.

Die Protokollierung auf Klassen-Basis stattfinden. Vorerst sollen
Protokollierungen als Stream ausgegeben werden. Pro Klasse muss also eine
\texttt{logging}-Instanz instanziert und mit dem entsprechenden Handler ausgestattet
werden. Um den Programmcode nicht unnötig wiederholen zu müssen, bietet sich
hierfür das Decorator-Pattern von Python
an\footnote{\url{https://www.python.org/dev/peps/pep-0318/}}.

Die Klasse zur Protokollierung soll also Folgendes tun:

\begin{itemize}
\item Einen Logger-Namen auf Basis des aktuellen Moduls und der aktuellen Klasse setzen
\begin{listing}[H]
\begin{minted}[,fontsize=\footnotesize,linenos,bgcolor=bashcodebg]{python}
logger_name = "%s.%s" % (cls.__module__, cls.__name__)
\end{minted}
\caption{\label{org187da4e}
Setzen des Logger-Names auf Basis des aktuellen Modules und Klasse.}
\end{listing}

\item Einen Stream-Handler nutzen
\begin{listing}[H]
\begin{minted}[,fontsize=\footnotesize,linenos,bgcolor=bashcodebg]{python}
stream_handler = logging.StreamHandler()
\end{minted}
\caption{\label{orgfc783ca}
Initialisieren eines Stream-Handlers.}
\end{listing}

\item Die Stufe der Protokollierung abhängig von der aktuellen Konfiguration setzen
\begin{listing}[H]
\begin{minted}[,fontsize=\footnotesize,linenos,bgcolor=bashcodebg]{python}
# TODO: Do this according to config.
stream_handler.setLevel(logging.DEBUG)
\end{minted}
\caption{\label{orgaf37712}
Setzen des \texttt{DEBUG} Log-Levels.}
\end{listing}

\item Protokoll-Einträge ansprechend formatieren
\begin{listing}[H]
\begin{minted}[,fontsize=\footnotesize,linenos,bgcolor=bashcodebg]{python}
# TODO: Set up formatter in debug mode only
formatter = logging.Formatter("%(asctime)s - %(levelname)-7s - %(name)s.%(funcName)s::%(lineno)s: %(message)s")
stream_handler.setFormatter(formatter)
\end{minted}
\caption{\label{org5f2e3ea}
Anpassung der Ausgabe von Protokoll-Meldungen.}
\end{listing}

\item Eine einfache Schnittstelle zur Protokollierung bieten
\begin{listing}[H]
\begin{minted}[,fontsize=\footnotesize,linenos,bgcolor=bashcodebg]{python}
cls.logger = logging.getLogger(logger_name)
cls.logger.propagate = False
cls.logger.addHandler(stream_handler)

return cls
\end{minted}
\caption{\label{org3f57ab3}
Nutzung des erstellten Stream-Handlers und Rückgabe der Klasse.}
\end{listing}
\end{itemize}

Nun kann die eigentliche Funktionalität implementiert werden.

\begin{listing}[H]
\begin{minted}[,fontsize=\footnotesize,linenos,bgcolor=bashcodebg]{python}
# -*- coding: utf-8 -*-

"""Module holding common helper methods."""

# System imports
import logging


def with_logger(cls):
    """Add a logger instance (using a steam handler) to the given class
    instance.

    :param cls: the class which the logger shall be added to
    :type  cls: a class of type cls

    :return: the class type with the logger instance added
    :rtype:  a class of type cls
    """

    <<logger-name>>
    <<logger-stream-handler>>
    <<logger-set-level>>
    <<logger-set-formatter>>
    <<logger-return-logger>>
\end{minted}
\caption{\label{orgd9a8781}
Das \texttt{common}-Modul und eine Methode zur Protokollierung in Klassen.}
\end{listing}


Der Decorator kann nun direkt auf die Klasse der QDE-Applikation angewendet
werden.

\begin{listing}[H]
\begin{minted}[,fontsize=\footnotesize,linenos,bgcolor=bashcodebg]{python}
@common.with_logger
class QDE(QApplication):
  """Main application for QDE."""

  <<app-class-body>>
\end{minted}
\caption{\label{org627e224}
Definition der Klasse \texttt{Application} mit dem \texttt{with\_logger}-Dekorator des \texttt{common}-Modules.}
\end{listing}

Damit die Protokollierung jedoch nicht nur via STDOUT in der Konsole statt
findet, muss diese entsprechend konfiguriert werden. Das \emph{logging}-Modul von
Python bietet hierzu vielfältige
Möglichkeiten.\footnote{\url{https://docs.python.org/3/library/logging.html}} So kann die
Protokollierung mittels der ``Configuration API'' konfiguriert werden. Hier
bietet sich die Konfiguration via Dictionary an. Ein Dictionary kann zum
Beispiel sehr einfach aus einer JSON-Datei generiert werden.

Die Haupt-Applikation soll die Protokollierung folgendermassen vornehmen:
\begin{itemize}
\item Die Konfiguration erfolgt entweder via externer JSON-Datei oder verwendet die
Standardkonfiguration, welche von Python mittels \texttt{basicConfig} vorgegeben
wird.
\item Als Name für die JSON-Datei wird \texttt{logging.json} angenommen.
\item Ist in den Umgebungsvariablen des Betriebssystems die Variable \emph{LOG\_CFG}
gesetzt, wird diese als Pfad für die JSON-Datei angenommen. Ansonsten wird
angenommen, dass sich die Datei \texttt{logging.json} im Hauptverzeichnis befindet.
\item Existiert die JSON-Konfigurationsdatei nicht, wird auf die
Standardkonfiguration zurückgegeriffen.
\item Die Protokollierung verwendet immer eine Protokollierungsstufe (Log-Level)
zum Filtern der verschiedenen Protokollnachrichten.
\end{itemize}

Die Haupt-Applikation nimmt also die Parameter \texttt{Pfad}, \texttt{Protokollierungsstufe}
sowie \texttt{Umgebungsvariable} entgegen.

Nun kann die eigentliche Umsetzung zur Konfiguration der Protokollierung
umgesetzt und der Klasse hinzugefügt werden.

\begin{listing}[H]
\begin{minted}[,fontsize=\footnotesize,linenos,bgcolor=bashcodebg]{python}
def setup_logging(self,
    default_path='logging.json',
    default_level=logging.INFO,
    env_key='LOG_CFG'
):
    """Setup logging configuration"""

    path = default_path
    value = os.getenv(env_key, None)

    if value:
        path = value

    if os.path.exists(path):
        with open(path, 'rt') as f:

            config = json.load(f)
            logging.config.dictConfig(config)
    else:
        logging.basicConfig(level=default_level)
\end{minted}
\caption{\label{org183d1e4}
Methode zum Initialisieren der Protokollierung der Applikation.}
\end{listing}


\begin{listing}[H]
\begin{minted}[,fontsize=\footnotesize,linenos,bgcolor=bashcodebg]{python}
# -*- coding: utf-8 -*-

"""Main application module for QDE."""

<<app-imports>>

<<app-class-definition>>
\end{minted}
\caption{Haupt-Modul und Einstiegspunkt der Applikation.}
\end{listing}

\begin{listing}[H]
\begin{minted}[,fontsize=\footnotesize,linenos,bgcolor=bashcodebg]{python}
<<app-system-imports>>

<<app-project-imports>>
\end{minted}
\caption{\label{org1c83ef7}
Definition der Importe des Haupt-Modules.}
\end{listing}

\begin{listing}[H]
\begin{minted}[,fontsize=\footnotesize,linenos,bgcolor=bashcodebg]{python}
# System imports
from   PyQt5.Qt import QApplication
from   PyQt5.Qt import QIcon
import logging
import os
\end{minted}
\caption{\label{orgb8719d4}
Importe von Python-eigenen Modulen im Haupt-Modul.}
\end{listing}

\begin{listing}[H]
\begin{minted}[,fontsize=\footnotesize,linenos,bgcolor=bashcodebg]{python}
# Project imports
from qde.editor.foundation import common
\end{minted}
\caption{\label{orgd5c3ab0}
Importe von selbst verfassten Modulen im Haupt-Modul.}
\end{listing}

\begin{listing}[H]
\begin{minted}[,fontsize=\footnotesize,linenos,bgcolor=bashcodebg]{python}
def __init__(self, arguments):
    """Constructor.

    :param arguments: a (variable) list of arguments, that are
                      passed when calling this class.
    :type  argv:      list
    """

    super(QDE, self).__init__(arguments)
    self.setWindowIcon(QIcon("assets/icons/im.png"))
    self.setApplicationName("QDE")
    self.setApplicationDisplayName("QDE")

    self.setup_logging()
\end{minted}
\caption{\label{org770f684}
Konstruktor des Haupt-Modules.}
\end{listing}

Der Konstruktor und die Methode zum Einrichten der Protokollierung werden
schliesslich der Klasse hinzugefügt.

\begin{listing}[H]
\begin{minted}[,fontsize=\footnotesize,linenos,bgcolor=bashcodebg]{python}
<<app-constructor>>

<<app-setup-logging>>
\end{minted}
\caption{\label{org1244001}
Hinzufügen des Konstruktors sowie der Methode zum Einrichten der Protokollierung zum Körper des Haupt-Modules.}
\end{listing}

\chapter{Working log}
\label{sec:orgdf17598}

\textit{<2017-02-20 Mon> } Initial set up and structuring of the document.

\chapter{Bibliography}
\label{sec:org87248af}

\printbibliography{}

\chapter{Appendix}
\label{sec:org1f6994c}

\section{Test cases}
\label{sec:org4cc26c0}

Zunächst wird jedoch der entsprechende Unit-Test definiert. Dieser instanziert
die Klasse und stellt sicher, dass sie ordnungsgemäss gestartet werden kann.

Als erster Schritt wird der Header des Test-Modules definiert.

\begin{listing}[H]
\begin{minted}[,fontsize=\footnotesize,linenos,bgcolor=bashcodebg]{python}
# -*- coding: utf-8 -*-

"""Module for testing QDE class."""
\end{minted}
\caption{\label{org5bba854}
Header des Test-Modules, \texttt{<<test-app-header>>}.}
\end{listing}

Dann werden die benötigen Module importiert. Es sind dies das System-Modul
\emph{sys} und das Modul \emph{application}, bei welchem es sich um die Applikation
selbst handelt. Das System-Modul \emph{sys} wird benötigt um der Applikation ggf.
Start-Argumente mitzugeben, also zum Beispiel:

\begin{listing}[H]
\begin{minted}[,fontsize=\footnotesize,linenos,bgcolor=bashcodebg]{bash}
python main.py argument1 argument2
\end{minted}
\caption{\label{org99f7f0d}
Aufruf des Main-Modules mit zwei Argumenten, \texttt{argument1} und \texttt{argument2}.}
\end{listing}

Der Einfachheit halber werden die Importe in zwei Kategorien unterteilt: Importe
von Pyhton-eigenen Modulen und Importe von selbst verfassten Modulen.

\begin{listing}[H]
\begin{minted}[,fontsize=\footnotesize,linenos,bgcolor=bashcodebg]{python}
<<test-app-system-imports>>

<<test-app-project-imports>>
\end{minted}
\caption{\label{org148d5ae}
Definition der Importe für das Modul zum Testen der Applikation.}
\end{listing}

\begin{listing}[H]
\begin{minted}[,fontsize=\footnotesize,linenos,bgcolor=bashcodebg]{python}
# System imports
import sys
\end{minted}
\caption{\label{org91fba36}
Importe von Python-eigenen Modulen im Modul zum Testen der Applikation.}
\end{listing}

\begin{listing}[H]
\begin{minted}[,fontsize=\footnotesize,linenos,bgcolor=bashcodebg]{python}
# Project imports
from qde.editor.application import application
\end{minted}
\caption{\label{orgc9a8396}
Importe von selbst verfassten Modulen im Modul zum Testen der Applikation.}
\end{listing}

Somit kann schliesslich getestet werden, ob die Applikation startet, indem diese
instanziert wird und die gesetzten Namen geprüft werden.

\begin{listing}[H]
\begin{minted}[,fontsize=\footnotesize,linenos,bgcolor=bashcodebg]{python}
def test_constructor():
    """Test if the QDE application is starting up properly."""
    app = application.QDE(sys.argv)
    assert app.applicationName() == "QDE"
    assert app.applicationDisplayName() == "QDE"
\end{minted}
\caption{\label{orgded7cc3}
Methode zum Testen des Konstruktors der Applikation.}
\end{listing}

Finally, one can merge the above defined elements to an executable test-module,
containing the header, the imports and the test cases (which is in this case
only a test case for testing the constructor).

\begin{listing}[H]
\begin{minted}[,fontsize=\footnotesize,linenos,bgcolor=bashcodebg]{python}
<<test-app-header>>

<<test-app-imports>>

<<test-app-test-constructor>>
\end{minted}
\caption{Modul zum Testen der Applikation.}
\end{listing}

Führt man die Testfälle nun aus, schlagen diese erwartungsgemäss fehl, da die
Klasse, und somit die Applikation, als solche noch nicht existiert. Zum jetzigen
Zeitpunkt kann noch nicht einmal das Modul importiert werden, da diese noch
nicht existiert.

\begin{listing}[H]
\begin{minted}[,fontsize=\footnotesize,linenos,bgcolor=bashcodebg]{bash}
python -m pytest qde/editor/application/test_application.py
\end{minted}
\caption{Aufruf zum Testen des Applkations-Modules.}
\end{listing}

Um sicherzustellen, dass die Protokollierung wie gewünscht funktioniert, wird
diese durch die entsprechenden Testfälle abgedeckt.

Der einfachste Testfall ist die Standardkonfiguration, also ein Aufruf ohne
Parameter.

\begin{listing}[H]
\begin{minted}[,fontsize=\footnotesize,linenos,bgcolor=bashcodebg]{python}
def test_setup_logging_without_arguments():
    """Test logging of QDE application without arguments."""
    app = application.QDE(sys.argv)
    root_logger = logging.root
    handlers = root_logger.handlers
    assert len(handlers) == 1
    handler = handlers[0]
\end{minted}
\caption{\label{org7077535}
Testfall 1 der Protkollierung der Hauptapplikation: Aufruf ohne Argumente.}
\end{listing}

Da obige Testfälle das \emph{logging}-Module benötigen, muss das Importieren der Module
entsprechend erweitert werden.

\begin{listing}[H]
\begin{minted}[,fontsize=\footnotesize,linenos,bgcolor=bashcodebg]{python}
import logging
\end{minted}
\caption{\label{orgbf36df9}
Erweiterung des Importes von System-Modulen im Modul zum Testen der Applikation.}
\end{listing}

Und der Testfall muss den Testfällen hinzugefügt werden.

\begin{listing}[H]
\begin{minted}[,fontsize=\footnotesize,linenos,bgcolor=bashcodebg]{python}
<<test-app-test-logging-default>>
\end{minted}
\caption{\label{org2d205f7}
Hinzufügen des Testfalles 1 zu den bestehenden Testfällen im Modul zum Testen der Applikation.}
\end{listing}

Auch hierfür werden wiederum zuerst die Testfälle verfasst.

\begin{listing}[H]
\begin{minted}[,fontsize=\footnotesize,linenos,bgcolor=bashcodebg]{python}
# -*- coding: utf-8 -*-

"""Module for testing common methods class."""

# System imports
import logging

# Project imports
from qde.editor.foundation import common


@common.with_logger
class FooClass(object):
    """Dummy class for testing the logging decorator."""

    def __init__(self):
        """Constructor."""
        pass

def test_with_logger():
    """Test if the @with_logger decorator works correctly."""

    foo_instance = FooClass()
    logger = foo_instance.logger
    name = "qde.editor.foundation.test_common.FooClass"
    assert logger is not None
    assert len(logger.handlers) == 1
    handler = logger.handlers[0]
    assert type(handler) == logging.StreamHandler
    assert logger.propagate == False
    assert logger.name == name
\end{minted}
\caption{\label{org72c710e}
Testfälle der Hilfsmethode zur Protokollierung.}
\end{listing}

\begin{minted}[]{bash}
python -m pytest qde/editor/foundation/test_common.py
\end{minted}


\section{Meeting minutes}
\label{sec:org27bdcac}

\subsection{Meeting mintutes 2017-02-23}
\label{sec:org2ba6021}

\begin{center}
\begin{tabular}{ll}
No.: & 01\\
Date: & 2017-02-23 13:00 - 13:30\\
Place: & Cafeteria, Main building, Berne University of applied sciences, Biel\\
Involved persons: & Prof. Claude Fuhrer (CF)\\
 & Sven Osterwalder (SO)\\
\end{tabular}
\end{center}

Kick-off meeting for the thesis.

\begin{enumerate}
\item Presentation and discussion of the current state of work
\label{sec:orgbf5024b}

\begin{itemize}
\item Presentation of the workflow. Emacs and Org-Mode is used to write the
documentation as well as the actual code. (SO)
\begin{itemize}
\item This is a very interesting approach. The question remains if the effort of
this method does not prevail the method of developing the application and
the documentation in parallel. It is important to reach a certain state of
the application. Also the report should not exceed around 80 pages. (CF)
\begin{itemize}
\item A decision about the used method is made until the end of this week. (SO)
\end{itemize}
\end{itemize}
\item The code will unit-tested using py.test and / or hypothesis. (SO)
\item Presentation of the structure of the documentation. It follows the schematics
of the preceding documentations. (SO)
\end{itemize}

\item Further steps / proceedings
\label{sec:org3f4272a}

\begin{itemize}
\item The expert of the thesis, Mr. Dubuis, puts mainly emphasis on the
documentation. The code of the thesis is respected too, but is clearly not the
main aspect. (CF)
\item Mr. Dubuis also puts emphasis on code metrics. Therefore the code needs to be
(automatically) tested and a coverage of at least 60 to 70 percent must be
reached. (CF)
\item A meeting with Mr. Dubuis shall be scheduled at the end of March or beginning
of April 2017. (CF)
\item The administrative aspects as well as the scope should be written until end of
March 2017 for being able to present them to Mr. Dubuis. (CF)
\item Mr. Dubuis should be asked if the publicly available access to the whole
thesis is enough or if he wishes to receive the particular status right before
the meetings. (CF)
\item Regularly meetings will be held, but the frequency is to be defined yet.
Further information follows per e-mail. (CF)
\item At the beginning of the studies, a workplace at the Berne University of
applied sciences in Biel was offered. Is this possibility still available?
(SO)
\begin{itemize}
\item Yes, that possibility is still available and details will be clarified and
follow per e-mail. (CF)
\end{itemize}
\end{itemize}

\item To do for the next meeting
\label{sec:org4295c23}

\begin{enumerate}
\item {\bfseries\sffamily DONE} Create GitHub repository for the thesis. (SO)
\label{sec:orgfb0506f}
\begin{enumerate}
\item {\bfseries\sffamily DONE} Inform Mr. Fuhrer about the creation of the repository. (SO)
\label{sec:org70cc49d}
\end{enumerate}

\item {\bfseries\sffamily DONE} Ask Mr. Dubuis by mail how he wants to receive the documentation. (SO)
\label{sec:org11e6cdf}
\begin{enumerate}
\item {\bfseries\sffamily TODO} Await answer of Mr. Dubuis (ED)
\label{sec:org0140f70}
\end{enumerate}

\item {\bfseries\sffamily DONE} Set up appointments with Mr. Dubuis (CF)
\label{sec:org7476328}
\begin{enumerate}
\item {\bfseries\sffamily TODO} Await answer of Mr. Dubuis (ED)
\label{sec:org53aa7fb}
\end{enumerate}

\item {\bfseries\sffamily DONE} Clarify possibility of a workplace at Berne University of applied sciences in Biel. (CF)
\label{sec:orgf2230ab}
\begin{enumerate}
\item A workplace was found at the RISIS laboratory and may be used instantly. (CF)
\label{sec:org900cda5}
\end{enumerate}

\item {\bfseries\sffamily DONE} Decide about the method used for developing this thesis. (SO)
\label{sec:orgf3283b5}
\begin{enumerate}
\item After discussions with a colleague the method of literate programming is
\label{sec:orge66b260}
kept. The documentation containing the literate program will although be
attached as appendix as it most likely will exceed 80 pages. Instead the
method will be introduced in the report and the report will be endowed
with examples from the literate program.
\end{enumerate}

\item {\bfseries\sffamily TODO} Describe procedure and set up a time schedule including milestones. (SO)
\label{sec:orgafa863b}
\end{enumerate}

\item Scheduling of the next meeting
\label{sec:org0e22edf}

\begin{itemize}
\item To be defined
\end{itemize}
\end{enumerate}
\end{document}